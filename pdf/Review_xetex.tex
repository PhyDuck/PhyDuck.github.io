\documentclass[14pt,oneside]{book}
%\usepackage{CJKutf8}
\usepackage{amsmath}
\usepackage{amsthm}
\usepackage{xeCJK}
\usepackage{mathrsfs}
\usepackage{amsfonts}
%\usepackage{dsfont}
\usepackage{graphicx}
\usepackage[CJKbookmarks=true]{hyperref}
%\usepackage{indentfirst}
\usepackage{physics}
\usepackage[a4paper]{geometry}
\geometry{left=2.6cm, right=2.6cm, top=3.7cm, bottom=2.1cm,}
\setlength{\footskip}{3cm}
\setlength{\parindent}{2em} %2em代表首行缩进两个字符
\usepackage{setspace} 
\usepackage{extarrows}
\usepackage{color}
%\usepackage[utf8]{inputenc}
\usepackage{indentfirst} 
\usepackage{float}
\hypersetup{colorlinks=true, linkcolor=blue}
\usepackage{subfigure}
\usepackage{esint}
\usepackage{listing}
\def\rr{{\mbox{$\resizebox{.09in}{.08in}{\includegraphics[trim= 1em 0 14em 0,clip]{ScriptR}}$}}}
\def\br{{\mbox{$\resizebox{.09in}{.08in}{\includegraphics[trim= 1em 0 14em 0,clip]{BoldR}}$}}}
%defining bold script r with a hat (unit vb*tor)
\newcommand{\hr}{\hat{\br}}
\def \J{\vb*{J}}
\def \i{\mathrm{i}}
\def \E{\vb*{E}}
\def \B{\vb*{B}}
\def \A{\vb*{A}}
\def \x{\vb*{\hat x}}
\def \y{\vb*{\hat y}}
\def \z{\vb*{\hat z}}
\def \laplace{\nabla^2{}}
\linespread{1.5}
%\begin{CJK}{UTF8}{gbsn}
\title{「电动力学」知识梳理}
\author{PhyDuck}
\date{November 2022}

\begin{document}
\begin{large}
\maketitle
\chapter*{Preface}

内容是电动力学知识梳理,我预计会侧重后面几篇(maybe…)。

之所以我的标题都是英文是因为目录不支持中文…emm 有点怪 不过也合理,毕竟我用的是Texpad里面的CJK这种古董(但是谁叫他能实时编译呢嘿嘿)

这篇梳理也是我自己复习的时候写的笔记,你可能会发现有些东西明明看起来很trivial 但是我还是推导了,这可能是因为我快忘了,想推一遍加深一下记忆(bushi),也许他能给我们带来一些启发,而一些比较复杂的内容我却没推导,可能是他们的推导过程很复杂,是因为我懒得推(bushi),其实是我们也许不需要去真的做这个事情,而他们的物理图像会更深入人心。

这篇笔记的电子版我会上传到百度网盘,和我的知识星球(可以用小程序下载pdf 感觉会比较方便 我也是刚发现还有这种东西,二维码在最后)。

因为事情也是比较多,可能会有一些typo,发现的朋友也可以在公众号私聊我,或者加我的微信(PhysicsDuck),我也是发现最近有很多不认识的朋友关注了我的公众号,如果大家有什么想要交流的可以随时通过这些办法给我发消息!谢谢大家的关注嘿嘿

就先写到这里啦~如果快的话一周能更两次(也许吧)剩下的等想到想说的再写吧哈哈哈哈!

~\\
\begin{flushright}
    \begin{tabular}{c}
        PhyDuck\\
        \date{November 2022}
    \end{tabular}
\end{flushright}
\tableofcontents
\setcounter{page}{0} 
\newpage



\pagenumbering{arabic}
\part{Static Fields}
\chapter{Electrostatics}\
\section{Coulomb's law}
库伦定律指出两个带电物体之间的受力遵循
\begin{equation}
  \vb*{F}=\frac{q_1 q_2}{\abs{\br}^2} \vb*{\hr}
\end{equation}
其中$\br\vb*{=r_1-r_2}$表示两个点电荷之间的间隔矢量,如果考虑一个在场点$r$试探电荷q与在$\vb*{r'}$的静止的场源电荷Q的作用,试探电荷受力为
\begin{equation}
    \vb*{F}=\frac{Qq}{\abs{\br}^2} \vb*{\hr}
\end{equation}这里的$\br=\vb*{r}-\vb*{r'}$从源点$\vb*{r'}$指向场点。

\section{Electric Field}
在孤立点电荷的公式中,我们可以发现一个不变量,在场点$\vb*{r}$处,试探电荷所受到的力仅与其电荷q有关,于是可以定义
\begin{equation}
\vb*{E}(\vb*{r})=\frac{\vb*{F}}{q}= \frac{Q}{\abs{\br}^2} \vb*{\hr}
\end{equation}
这个量只与空间坐标$\vb*{r}$有关(场源电荷确定的情况下),可以表征一个孤立点电荷产生的场在空间中的属性,定义其电场强度。

可以证明,电场强度服从叠加原理
\begin{equation}
  \vb*{E}(\vb*{r})=\sum_i \frac{Q_i}{\abs{\br_i}^2} \vb*{\hr_i}
\end{equation}
在电荷连续分布时,求和可以写成积分
\begin{equation}
  \sum_i Q_i=\int \lambda \dd l=\int \sigma \dd S =\int \rho \dd V
\end{equation}
\section{Properties of Electric Field}
\textbf{亥姆霍兹定理}告诉我们,任一矢量场由它的散度、旋度和边界条件唯一确定,通过解微分方程,代入边值条件我们就可以得到场的信息。所以在研究一个矢量场的时候,知道其散度和旋度是有意义的。
\subsection{Divergence}
电场的散度可以由矢量场的高斯定理得到:
\begin{equation}
  \iiint\div{\vb*{E}} \dd V=\oiint \vb*{E}\cdot\dd \vb*{S} =\sum \frac{Q_{in}}{\epsilon_0}=\iiint\frac{\rho}{\epsilon_0}\dd V
\end{equation}
所以
\begin{equation}
  \boxed{\div{\vb*{E}}=\frac{\rho}{\epsilon_0}}
\end{equation}
\subsection{Curl}
同样的,通过Stokes定理可以得到
\begin{equation}
  \iint \curl{\vb*{E}}\cdot\dd \vb*{S}=\oint \vb*{E}\cdot\dd \vb*{l}=0
\end{equation}
于是
\begin{equation}
 \boxed{ \curl{\vb*{E}}=0}
\end{equation}

\section{Potential}
由于$\curl{\vb*{E}}=0$,而梯度场是无旋的,因此定义
\begin{equation}
  \vb*{E}=-\nabla V
\end{equation}
V被称为电势,可以将(1.10)写作积分形式,即
\begin{equation}
  V(\vb*{r_a})-V(\vb*{r_b})=\int_{\vb*{r_a}}^{\vb*{r_b}}\vb*{E}\cdot\dd \vb*{l}
\end{equation}
由于(1.8)的后式,这个积分不依赖于路径(注意一般的线积分是依赖于始末点和路径的)。由于场才是唯一的、实在的,所以电势的选择具有一定任意性(你可以在1.10的V的基础上加上一个常数或标量,电场仍然不变)。你可以选择一个电势零点(如果你的电荷在无穷远处没有分布,通常可以选择无穷远点),这样
\begin{equation}
   \boxed{ V(\vb*{r})=\int_{\vb*{r}}^{\infty}\vb*{E}\cdot\dd \vb*{l}}
\end{equation}

\subsection{Poisson Equation}
将(1.10)式带入(1.7)式,可以得到泊松方程:
\begin{equation}
  \nabla^2{V}=-\frac{\rho}{\epsilon_0}
\end{equation}
这个关于势的方程结合了场的散度与旋度的性质,所以解静电场问题时,使用(1.13)式加以边界条件即可求出静电场的标量势,进而可以求出静电场的场强。(当然很多时候,在有某种对称性的前提下,使用Gauss定理积分会简单的多)

泊松方程的解是:
\begin{equation}
 \boxed{ V=\frac{1}{4\pi\epsilon_0}\sum \frac{q_i}{\rr_i}}
\end{equation}
同样的,可以应用式(1.5)将连续分布的电荷转化为积分。
\section{Boundary Condition}
研究场的边界条件是有助于解决某些问题的(比如导体),首先考虑垂直于界面的电场变化:
\begin{figure}[H]
\centering  %图片全局居中
\subfigure[垂直方向]{
\label{Fig.sub.1}
\includegraphics[width=0.45\textwidth]{ebc1}}
\subfigure[平行方向]{
\label{Fig.sub.2}
\includegraphics[width=0.45\textwidth]{ebc2}}
\caption{边界条件示意图}
\label{Fig.main}
\end{figure}
如\ref{Fig.sub.1}所示,在边界某处做一高$\epsilon\rightarrow0$,底面积A也很小(为了研究垂直方向的电场变化,这个尺度应当要比)的立方体高斯面,由高斯定律
\begin{equation}
  E^\perp_{above}A-E^\perp_{below}A=\frac{\sigma A}{\epsilon_0}
\end{equation}
即
\begin{equation}
  、E^\perp=\frac{\sigma}{\epsilon_0}
\end{equation}
接着考虑平行于界面方向的电场,如\ref{Fig.sub.2}所示,作矩形回路积分,注意同样的有$\epsilon\ll l\ll 1$
\begin{equation}
  E^\parallel_{above}l-E^\parallel_{below}l=0
\end{equation}
即
\begin{equation}
  E^\parallel_{above}=E^\parallel_{below}\quad\text{or}\quad \Delta E^\parallel=0
\end{equation}
综合(1.16)和(1.18),可以给出电矢量的边界条件
\begin{equation}
 \boxed{\Delta \vb*{E}=\frac{\sigma}{\epsilon_0}\vb*{\hat{n}}}
\end{equation}
考虑到$\vb*{E}=-\nabla V$与$V=\int \vb*{E}\cdot \dd \vb*{l}$,所以
\begin{equation}
  \Delta V =\int_{-\frac{\epsilon}{2}}^\frac{\epsilon}{2}\vb*{E}\cdot \dd \vb*{l}\xlongequal[]{\epsilon\rightarrow0}0
\end{equation}
  \begin{equation}
\frac{\partial V_{{above }}}{\partial n}-\frac{\partial V_{ {below }}}{\partial n}=-\frac{1}{\epsilon_0} \sigma
\end{equation}
\section{Energy}

在考虑完场的性质后,下一个值得思考的问题是这个场携带的能量有多少。电场所带的能量是在一个什么都没有(在实际计算中是从只存在一个点电荷出发,由于一个点电荷的自能会是无穷大)的空间创造出某一电荷构型所需要做的功,可以写作
\begin{equation}
\begin{aligned}
  W&=\frac{1}{4\pi\epsilon_0}\sum_i\sum_{j>i}\frac{q_i q_j}{\rr_{ij}}\\
  &=\frac{1}{2}\frac{1}{4\pi\epsilon_0}\sum_i \sum_{j\neq i}\frac{q_i q_j}{\rr_{ij}}(\frac{1}{2}\text{是由于势能的重复计算}	)\\
  &=\sum_i q_i U_i
  \end{aligned}
\end{equation}
其中,$U_i$为其他所有电荷在该处产生的电势。

对于电荷连续分布的情况
\begin{equation}
  W=\frac{\epsilon_0}{2}\int_{\text{全空间}} E^2 \dd V
\end{equation}
\section{Conductor}
\subsection{Properties}
导体是一种内部有无穷多可以自由移动的电子的物体,其有如下性质
\begin{enumerate}
  \item \textbf{静电屏蔽}
  
  由于内部存在无穷多自由电子,则对于外部给定静电场E,内部电子会移动产生内电场E'直至与外电场抵消。
  \item \textbf{内部无净体自由电荷,他们分布在导体的外表面}
  
  考虑到$\div{\vb*{E}}=\frac{\rho}{\epsilon_0}$,$\vb*{E}=0$所以内部无没有被抵消的体自由电荷分布,他们会分布在导体的外表面
  \item \textbf{导体是等势体}
  \item \textbf{$\vb*{E}$垂直于导体表面}
\end{enumerate}
\subsection{Surface Charge}
由于导体内部不存在电场,且在导体表面,电场垂直于导体,由边界条件(1.19),
\begin{equation}
  \vb*{E}_{\text{surface}}=\frac{\sigma}{\epsilon_0}\vb*{\hat{n}}
\end{equation}
或
\begin{equation}
  -\pdv{V}{n}\epsilon_0=\sigma_{\text{surface}}
\end{equation}
\section{Electric Field in Matter}
\subsection{Polarization of Dielectrics}
在真空中的静电学的基础上,我们可以探究静电场在介质(绝缘体)中的性质。

与导体不同,绝缘体内并没有(有限)个可以自由移动的电子,介质中每个分子可以抽象为一个电偶极子,有偶极矩
\begin{equation}
  \vb*{p}=q\vb*{d}
\end{equation}
$\vb*{d}$是从负电荷指向正电荷的矢量,其在电场作用下,质心位置不变,而$\vb*{p}$会发生一个转动,转向与电场方向相同。这将产生一个累积效应:在介质中所有的偶极子均指向电场的方向,等效起来就像只有垂直于电场线的边界面上分布着正、负的束缚电荷,这也是面束缚电荷的产生原因。我画了一张图,也许能直观的说明这个问题。
\begin{figure}[H]
\centering
  \includegraphics[width=16cm]{Polarization.pdf}
  \caption{电介质的极化}
\end{figure}
当然,这个图像意味着均匀极化,非均匀极化会导致一个体束缚电荷密度。
\subsection{Potential and Bound Charge}
设极化强度\begin{equation}
  \vb*{P}(\vb*{r'})=\frac{\sum \vb*{p}}{\dd \tau}
\end{equation}
考虑偶极子的电势(设场点距离源点较远)
\begin{equation}
  V_{\vb*{p}}(\vb*{r})=\frac{1}{4\pi\epsilon_0}\frac{\hr\cdot\vb*{p}(\vb*{r'})}{\rr^2}
\end{equation}
于是
\begin{equation}
\begin{aligned}
	    V(\vb*{r})&=\iiint\frac{1}{4\pi\epsilon_0}\frac{\hr\cdot\vb*{P}(\vb*{r'})}{\rr^2}\dd \tau	\\
	    &=\underbrace{\frac{1}{4\pi\epsilon_0}\oiint_S \frac{\vb*{P}(\vb*{r'})}{\rr}\cdot\dd \vb*{S}}_{\text{边界上面束缚电荷的贡献}}{\color{red} -}\underbrace{\frac{1}{4\pi\epsilon_0}\iiint_\Omega \frac{\div{\vb*{P}(\vb*{r'})}}{\rr}\dd V}_{\text{内部体束缚电荷贡献}}
\end{aligned}
\end{equation}
可以看出,电势存在两部分,一部分是一个面电荷密度分布产生的电势,一部分是体电荷密度分布产生的电势,我们可以定义
\begin{equation}
  \begin{aligned}
	  \sigma_b&=\vb*{P}\cdot \hat{\vb*{n}}	\\
	  \rho_b&={\color{red} -}\vb*{\nabla}\cdot \vb*{P}
\end{aligned}
\end{equation}
有了体电荷密度的定义,我们就可以建立有介质时的高斯定理了:
\begin{equation}
  \div{\vb*{E}}=\frac{\rho}{\epsilon_0}=\frac{\rho_b+\rho_f}{\epsilon_0}=-\frac{1}{\epsilon_0}\div{\vb*{P}}+\frac{\rho_f}{\epsilon_0}
\end{equation}
于是可以定义
\begin{equation}
  \vb*{D}=\epsilon_0 \vb*{E}+\vb*{P}
\end{equation}
有
\begin{equation}
  \boxed{\div{\vb*{D}}=\rho_f}
\end{equation}
$\vb*{D}$的旋度不一定为0,这依赖着你的极化强度:
\begin{equation}
  \curl{\vb*{D}}=\curl{\vb*{P}}
\end{equation}
\subsection{Linear Dielectrics}
顺便说一下,对于线性介质来说
\begin{equation}
  \vb*{P}=\epsilon_0\chi_e \vb*{E}
\end{equation}
于是
\begin{equation}
  \vb*{D}=\epsilon_0(\chi_e+1)\vb*{E}=\epsilon_0\epsilon_r \vb*{E}
\end{equation}
\subsection{Boundary Condition}
边界条件的导出是类似的,这里就不再写一遍了,直接给出结论
\begin{equation}
  \begin{aligned}
	 \Delta \vb*{D}^\perp &= \sigma_f 	\\
	 \Delta \vb*{D}^\parallel &=\Delta \vb*{P}^\perp
\end{aligned}
\end{equation}

\chapter{Potentials}
\section{Laplace's Equation}
我们希望研究的区域经常只是没有电荷分布的空间(可能是源点产生的场在真空或介质中的分布,如我们后面要研究的电磁波问题),此时式(1.13)的Poisson方程将退化为Laplace方程:
\begin{equation}
  \nabla^2{V}=0
\end{equation}
Laplace方程的若干性质对我们的求解是有帮助的。
\subsection{Properties of Laplace's Equation}
\begin{enumerate}
  \item \textbf{Laplace方程在某处的解可以写作其周围解的平均值}
  
  这意味着在一维情况下
  \begin{equation}
  V(x)=\frac{V(x-a)+V(x+a)}{2}
\end{equation}
二维情况下
\begin{equation}
  V(x,y)=\frac{1}{2\pi a}\oint V \dd l
\end{equation}
环路积分路径是以$(x,y)$为中心,半径为a的圆,而三维情况下
\begin{equation}
  V(x,y,z)\frac{1}{4\pi a^2}\oiint V\dd S
\end{equation}
曲面积分区域是以$(x,y,z)$为中心,半径为a的球。

有意思的是,这个性质\textbf{并不限定a的大小},并不需要积分路径处于邻域内。
\item \textbf{Laplace方程的解不存在局域的极大或极小值,极大,极小值仅存在于边界上}

这个性质可以由性质1得出,因为在非边界上的点的V,都可以写作其周围的V的平均值,这就意味着非边界上的点需要介于边界的最大值与最小值之间(或均相同)。
\end{enumerate}

\subsection{Uniqueness Theorems}
根据偏微分方程的性质,对于不含时的微分方程,其定解问题的解将由其边界条件来确定。所以在这一部分,我们想要解决的问题是:什么样的边界条件可以唯一地确定Laplace方程的解
\begin{enumerate}
  \item{\textbf{The First Uniqueness Theorem}}
  
  \qquad 第一唯一性定理是指,若一个区域$\mathcal{V}$的边界$\mathcal{S}$上的电势确定了,则这个区域内的电势(Laplace方程的解)将被唯一确定。
  \begin{proof}
  	若存在两个解$V_1,V_2$均对应于同一边界条件,即在区域$\mathcal{V}$内,满足
  	\begin{equation}
  \nabla^2 V_1=0\quad,\quad \nabla^2 V_2 =0
	\end{equation}
由于线性微分方程的性质,$V_1-V_2$也是该区域Laplace方程的解
\begin{equation}
  \nabla^2( V_1-V_2)=0
\end{equation}
这意味着
\begin{equation}
  \nabla^2 V_1=\nabla^2 V_2
\end{equation}
由于$V_1$和$V_2$遵循相同的边界条件,即在边界上有$V_1=V_2$,于是在边界上$V_3=0$,
又因为Laplace的极大、极小值在边界处取到,所以
\begin{equation}
  {V_3^{Max}}={V_3^{min}}=0\Rightarrow V_3=0(\text{Everywhere in region $\mathcal{V}$})
\end{equation}
即
\begin{equation}
  V_1=V_2(\text{Everywhere in region $\mathcal{V}$})
\end{equation}
  \end{proof}
 \qquad 第一唯一性定理有一个推论:
  若一个区域$\mathcal{V}$的边界$\mathcal{S}$上的电势以及内部的电荷密度分布$\rho$确定了,则这个区域内的电势将被唯一确定。(可以通过对泊松方程重复以上的证明来得到)




\item{\textbf{The Second Uniqueness Theorem}} 

\qquad 有导体存在时,我们可以引入第二唯一性定理。若一个区域$\mathcal{V}$内部的各个导体带电量$Q_i$确定,且区域内部的电荷密度分布确定,这个区域的电场将被唯一确定。
\begin{proof}
	
\begin{figure}[H]
	\centering
  \includegraphics[width=8cm]{Uniqueness Theorems.png}
  \caption{第二唯一性定理}
\end{figure}
假设$Q,\rho$有对应的两个可能的场$\vb*{E_1,E_2}$,在每个导体周围(内边界),做Gauss面如图,由Gauss定理,有
\begin{equation}
 \oiint_{S_i}\vb*{E_1}\cdot\dd \vb*{S}=\frac{Q_i}{\epsilon_0},\oiint_{S_i}\vb*{E_2}\cdot\dd \vb*{S}=\frac{Q_i}{\epsilon_0}
\end{equation}
在外边界上,
\begin{equation}
   \oiint_{\mathcal{S}}\vb*{E_1}\cdot\dd \vb*{S}=\frac{Q_{all}}{\epsilon_0},   \oiint_{\mathcal{S}}\vb*{E_2}\cdot\dd \vb*{S}=\frac{Q_{all}}{\epsilon_0}
\end{equation}
在内外边界之间存在$\rho$的区域,有
\begin{equation}
  \div{\vb*{E_1}}=\div{\vb*{E_2}}=\frac{\rho}{\epsilon_0}
\end{equation}
所以$\vb*{E_3=E_1-E_2}$遵循:
\begin{equation}
  \oiint_{\text{边界}}\vb*{E_3}\cdot\dd \vb*{S}=0,\vb*{\nabla}\cdot \vb*{E_3}=0(\text{边界之间})
\end{equation}
考虑一个散度:$\div{V_3\vb*{E_3}}$
\begin{equation}
  \div{V_3\vb*{E_3}}=V_3\div{\vb*{E_3}}+\vb*{E_3}\cdot\nabla V_3
\end{equation}
考虑到(2.13),有
\begin{equation}
  \div{V_3\vb*{E_3}}=-E_3^2
\end{equation}
在$\mathcal{V}$内积分:
\begin{equation}
  \iiint_\mathcal{V} \div{V_3\vb*{E_3}}\dd V =\oiint_\mathcal{S} V_3 \vb*{E_3}\cdot\dd \vb*{S}=\iiint_{\mathcal{V}} \abs{\vb*{E_3}}^2 \dd V
\end{equation}
考虑到内边界是导体,是等势面,外边界可以延伸到无穷远处(V=0),当然,在有限远处由于第一唯一性定理,$V_1=V_2$,所以$V_3=0$,又因为$E_3$的闭合面积分为0,式(2.13),可以导出
\begin{equation}
  \iiint_{\mathcal{V}} -\abs{\vb*{E_3}}^2 \dd V= V_3\oiint_\mathcal{S} \vb*{E_3}\cdot\dd \vb*{S}=0\Rightarrow \abs{\vb*{E_3}}=0\Rightarrow \vb*{E_1=E_2}
\end{equation}
\end{proof}
\end{enumerate}

\section{The Method of Images}
在唯一性定理的基础上,我们可以引入镜像法来求解带有一定特殊边值条件的静电问题。在这里就只回顾一下最基本的镜像问题(无限大平板)。
\begin{figure}[H]
\centering  %图片全局居中
\subfigure[Problem]{
\label{Fig2.sub.1}
\includegraphics[width=0.35\textwidth]{Image}}
\subfigure[Solution]{
\label{Fig2.sub.2}
\includegraphics[width=0.35\textwidth]{Image_Solution}}
\caption{镜像问题示意图}
\label{Fig2.main}
\end{figure}
如\ref{Fig2.sub.1}所示,在$(0,0,d)$处放置一电荷量为$	q$的点电荷,在$z=0$平面有一无限大导体板,其电势为0,我们要求的是$z>0$区域内的电势。直接求解这个问题有些困难,而第一唯一性定理告诉我们,边界条件确定时,电势就唯一确定了,受此启发,我们可以考虑一种转换的方式:在$(0,0,-d)$处放置一个电荷量为$-q$的点电荷,在这两个点电荷的作用下,$z=0$平面的$V=0$,这与原问题给出了相同的边界条件,而这个问题我们是熟悉的,其解为
\begin{equation}
  V(x,y,z)=\frac{q}{4\pi\epsilon_0}(\frac{1}{\rr_+}-\frac{1}{\rr_-})
\end{equation}

\section{Separation of Variables }

在有确定边值条件(与初值条件)的偏微分方程的求解中,分离变量法是一个有力的工具,直角坐标系的分离变量法是简单的,我打算一笔带过,这一节主要来说说球坐标系。
\subsection{Cartesian Coordinates}
其实我一直想知道Cartesian为什么翻译过来是笛卡尔,在写这个标题的时候查了一下,终于了解到原来是René Descartes的翻译,Cartesius是他的拉丁名,而Cartesian是他的英文(写法),真是神奇哈哈

直角坐标系下的二维Laplace方程可以写为:
\begin{equation}
  \pdv[2]{V}{x}+\pdv[2]{V}{y}=0
\end{equation}
令$V(x,y)=X(x)Y(y)$,方程转化为
\begin{equation}
  X''Y+XY''=0
\end{equation}
同除$V$,可以令
\begin{equation}
  \frac{X''}{X}=-\frac{Y''}{Y}=k^2
\end{equation}
这就将原来的偏微分方程转化为了两个常微分方程
\begin{align}
	  X''-k^2X=0	\\
	  Y''+k^2Y=0
\end{align}
这两个微分方程的通解为
\begin{align}
	  X=A\sin(k x+\delta_1)	\\
	  Y=B\mathrm{e}^{ky}+C\mathrm{e}^{-ky}
\end{align}
于是V可以写成一个乘积的形式,
\begin{equation}
  V=D\sin(k x+\delta_1)(\mathrm{e}^{ky}+\mathrm{e}^{-ky})
\end{equation}
当然这里还有一些系数的问题,k的选取也很可能不是随意的,最终,你可能还需要借助广义傅里叶级数来进行展开等等,总之,这依赖于边值条件,一个分离变量法的完整例子我将在球坐标系中给出。

\subsection{Spherical Coordinates}
在球坐标系下的Laplace方程为
\begin{equation}
  \frac{1}{r^2}\pdv{}{r}\left(r^2\pdv{V}{r}\right) +\frac{1}{r^2\sin\theta}\pdv{}{\theta }\left(\sin\theta\pdv{V}{\theta}\right)+\frac{1}{r^2\sin^2\theta}\pdv[2]{V}{\phi}=0
\end{equation}
可以分为两种情况,一种是轴对称势,即$V=V(r,\theta)$,与$\phi$无关,而另一种是普遍情况,先来看轴对称势
\begin{enumerate}
  \item \textbf{轴对称势}
 
  此时,V与$\phi$无关,于是Laplace方程改写为
  \begin{equation}
  \frac{1}{r^2}\pdv{}{r}\left(r^2\pdv{V}{r}\right) +\frac{1}{r^2\sin\theta}\pdv{}{\theta }\left(\sin\theta\pdv{V}{\theta}\right)=0
\end{equation}
分离变量,令$V(r,\theta)=R(r)\Theta(\theta)$,得到
\begin{equation}
\begin{aligned}
 \Theta \frac{1}{r^2}\pdv{}{r}\left(r^2R'\right)+R\frac{1}{r^2\sin\theta}\pdv{}{\theta }\left(\sin\theta\Theta'\right)=0\\
 \frac{1}{R}(2rR'+r^2R'')+\frac{1}{\Theta}\frac{1}{\sin \theta}\pdv{}{\theta }\left(\sin\theta\Theta'\right)=0
\end{aligned}
\end{equation}
可以令
\begin{align}
\begin{cases}
	\frac{1}{R}(2rR'+r^2R'')=l(l+1)\\
	\frac{1}{\Theta}\frac{1}{\sin \theta}\pdv{}{\theta }\left(\sin\theta\Theta'\right)=-l(l+1)
\end{cases}
\end{align}
对于径向方程,可以写成
	\begin{equation}
     R''(r)+\frac{2}{r} R'(r)-\frac{l(l+1)}{r^2}R(r)=0
	\end{equation}
	令$\chi(r)=rR(r)$,方程转化为
	\begin{equation}
  r^2\chi''(r)-l(l+1)\chi(r)=0
\end{equation}
由于二阶导的$r^2$系数,可以猜想出$\chi$有r的幂次形式,即
\begin{equation}
  \chi=r^S
\end{equation}
解得
\begin{equation}
  S=-l\quad  \text{or}\quad l+1
\end{equation}
所以通解是
\begin{equation}
  R(r)=Ar^l+\frac{B}{r^{l+1}}
\end{equation}
对于$\theta$的方程,可以整理为

\begin{equation}
\frac{\dd}{\dd \theta}\left(\sin \theta \frac{\dd \Theta}{\dd \theta}\right)=-l(l+1) \sin \theta \Theta
\end{equation}
他的解是勒让德级数(我就不解了)
\begin{equation}
\Theta(\theta)=P_l(\cos \theta)
\end{equation}
勒让德级数可以由罗德里格斯方程给出
\begin{equation}
P_l(x) \equiv \frac{1}{2^l l !}\left(\frac{\dd}{\dd x}\right)^l\left(x^2-1\right)^l
\end{equation}
他的前几项是
\begin{equation}
\begin{aligned}
&P_0(x)=1 \\
&P_1(x)=x \\
&P_2(x)=\left(3 x^2-1\right) / 2 \\
&P_3(x)=\left(5 x^3-3 x\right) / 2 \\
&P_4(x)=\left(35 x^4-30 x^2+3\right) / 8 \\
&P_5(x)=\left(63 x^5-70 x^3+15 x\right) / 8
\end{aligned}
\end{equation}
所以球坐标系的分离变量法给出的解是
\begin{equation}
V(r, \theta)=\sum_{l=0}^{\infty}\left(A_l r^l+\frac{B_l}{r^{l+1}}\right) P_l(\cos \theta)
\end{equation}
\item{\textbf{一般情况}}

一般的,球坐标系下的Laplace方程为
\begin{equation}
  \frac{1}{r^2}\pdv{}{r}\left(r^2\pdv{V}{r}\right) +\frac{1}{r^2\sin\theta}\pdv{}{\theta }\left(\sin\theta\pdv{V}{\theta}\right)+\frac{1}{r^2\sin^2\theta}\pdv[2]{V}{\phi}=0
\end{equation}
对于这个方程,我们仍然可以重复上面的方法,令$V(r,\theta,\phi)=R(r)Y(\theta,\phi)$,仍然取分离变量常数l(l+1)
此时方程转化为:
\begin{equation}
  \frac{1}{R}\frac{\rm d}{\mathrm{d}r}\left(r^2\frac{\mathrm{d}R}{\mathrm{d}r}\right)=-\frac{1}{sin\theta Y}\frac{\partial}{\partial \theta}\left(\sin\theta\frac{\partial Y}{\partial\theta}\right)-\frac{1}{Y}\frac{1}{\sin^2\theta}\frac{\partial^2 Y}{\partial \phi^2}=l(l+1)\\
\end{equation}
此时径向方程的解仍然为
\begin{equation}
  R(r)=Ar^l+\frac{B}{r^{l+1}}
\end{equation}
对$Y(\theta,\phi)$进一步分离变量:令$Y(\theta,\phi)=\Theta(\theta)\Phi(\phi)$,于是得到
\begin{equation}
  \frac{\sin{\theta}}{\Theta}(\cos\theta\Theta'+\sin\theta\Theta'')+l(l+1)\sin^2\theta+\frac{1}{\Phi}\Phi''=0
\end{equation}
令分离变量常数为$m^2$,做换元$x=\cos{\theta}$得到以下两常微分方程
\begin{align}
		\frac{\mathrm{d}^2 \Phi}{\mathrm{d} \phi^2}&+m^2 \Phi=0 \\
\left(1-x^2\right) \frac{\mathrm{d}^2 \Theta}{\mathrm{d} x^2}-2 x \frac{\mathrm{d} \Theta}{\mathrm{d} x}+&\left[l(l+1)-\frac{m^2}{1-x^2}\right] \Theta=0
\end{align}
(2.45)的解是简单的:
\begin{equation}
  \Phi_m=C\mathrm{e}^{im\phi}
\end{equation}
而(2.46)为连带勒让德方程,与(2.36)的勒让德方程相比,其多出一个系数m,如果仔细对比,可以发现勒让德方程求导m次就是连带勒让德方程,他的解是连带勒让德级数
\begin{equation}
  \Theta_{lm}=D P_l^m(\cos \theta)
\end{equation}
于是
\begin{equation}
 {{Y}_{lm}}\left( \theta ,\varphi \right)=E{{P}_{lm}}\left( \cos \theta \right){{e}^{im\varphi }},m=-l,-l+1,\cdots,l-1,l\\
\end{equation}
,这实际上是球谐函数(在量子力学中还应当归一化,电动力学中也需要根据边界条件来确定系数)最终,一般情况的球坐标系下的Laplace方程解为
\begin{equation}
 V(r,\theta,\phi)=  \sum_l\left[ \left(Ar^l+\frac{B}{r^{l+1}}\right)\sum_m Y_{lm}(\theta,\phi)\right]
\end{equation}


\end{enumerate}
下面来看一个书上的例题:在一匀强电场中放置一金属球,求金属球外的电势。
\begin{figure}[H]
\centering
  \includegraphics[width=5cm]{Metal Ball.png}
\end{figure}
很显然,这里的场和电势都是关于z轴对称的,我们可以利用式$(2.40)$来求解,现在唯一需要做的是找到边值条件来确定$A,B$和我们需要$P_l(\cos{\theta})$的哪些项。

导体球是一个等势体,我们可以令其势能为0,即$V(R,\theta,\phi)=0$。另一方面,在$z=0$平面附近,电场线实际上是与$z=0$面垂直的,这就导致了$z=0$面也是个等势面,而在远处,诱导电荷产生的场的作用很微弱,可以认为电场线平行于z轴,是一个匀强电场,所以
\begin{equation}
  V(r,\theta)=-Er\cos\theta(r\gg R)
\end{equation}
将这两个边界条件分别代入$(2.40)$,首先考虑r=R时电势为0,即
\begin{equation}
0=\sum_{l=0}^{\infty}\left(A_l R^l+\frac{B_l}{R^{l+1}}\right) P_l(\cos \theta)
\end{equation}
由于$P_l(\cos{\theta})$是不等于0的,只能要求
\begin{equation}
 A_l R^l=-\frac{B_l}{R^{l+1}}
\end{equation}
即
\begin{equation}
  B_l=-A_l R^{2l+1}
\end{equation}
整理为
\begin{equation}
V(r, \theta)=\sum_{l=0}^{\infty} \left(A_l r^l-A_l \frac{R^{2 l+1}}{r^{l+1}} \right)P_l(\cos \theta)
\end{equation}
考虑到$ V(r,\theta)=-Er\cos\theta(r\gg R)$,方程可以写为
\begin{equation}
  -Er\cos\theta=\sum_{l=0}^{\infty} \left(A_l r^l-A_l \frac{R^{2 l+1}}{r^{l+1}} \right)P_l(\cos \theta)
\end{equation}
考虑到左边只有$\cos{x}$一次方项,结合$P_0(\cos x)=1,P_1(\cos x)=\cos x,P_2(\cos x)=\frac{1}{2}(3\cos^2 x-1)$,我们需要的应当是$P_1(\cos x)$,将l=1代入,有
\begin{equation}
  -Er\cos\theta=\left(A_1 r-A_1 \frac{R^{3}}{r^{2}}\right)\cos\theta\xlongequal[]{r\gg R}A_1r\cos\theta
\end{equation}
所以可以得到$A_1=-E$
将上面的结果代入$(2.55)$,可以得到金属球外的电势
\begin{equation}
  V(r,\theta)=E\left(\frac{R^{3}}{r^{2}}-r\right)\cos\theta
\end{equation}
进一步,我们还可以求得感应电荷,借助我们在导体部分得到的式(1.25),即
\begin{equation}
  \sigma=-\epsilon_0\pdv{V}{n}
\end{equation}
在这个问题中,
\begin{equation}
\begin{aligned}
  \sigma&=-\epsilon_0\eval{\pdv{V}{r}}_{r=R}\\
  &=-\epsilon_0 E(-2-1)\cos \theta\\
  &=3\epsilon_0 E \cos\theta
  \end{aligned}
\end{equation}

\section{Multipole Expansion}
我们在第一章中就给出场点的电势:
\begin{equation}
  V(\vb*{r})=\frac{1}{4\pi\epsilon_0}\iiint_\mathcal{V}\frac{\rho(\vb*{r'})}{\rr_i}\dd \tau'
\end{equation}
\begin{figure}[H]
\centering
  \includegraphics[width=5cm]{Multipole.png}
\end{figure}

多级展开给了我们一种距离源点较远处的电势近似方法,在$r \gg r'$时,
\begin{equation}
\frac{1}{\rr}=\frac{1}{r} \sum_{n=0}^{\infty}\left(\frac{r^{\prime}}{r}\right)^n P_n(\cos \alpha)
\end{equation}
于是
\begin{equation}
V(\vb*{r})=\frac{1}{4 \pi \epsilon_0} \sum_{n=0}^{\infty} \frac{1}{r^{(n+1)}} \int\left(r^{\prime}\right)^n P_n(\cos \alpha) \rho\left(\vb*{r}^{\prime}\right) d \tau^{\prime},
\end{equation}
也可以一项一项把这个求和分开:
\begin{equation}
V(\vb*{r})=\frac{1}{4 \pi \epsilon_0}\left[\frac{1}{r} \int \rho\left(\vb*{r}^{\prime}\right) d \tau^{\prime}+\frac{1}{r^2} \int r^{\prime} \cos \alpha \rho\left(\vb*{r}^{\prime}\right) d \tau^{\prime}+\cdots\right]
\end{equation}
首先看第一项
\begin{equation}
V_{\text{单}}(\vb*{r})=\frac{1}{4 \pi \epsilon_0} \frac{Q}{r}
\end{equation}
其中,$Q=\int \rho\left(\vb*{r}^{\prime}\right) d \tau^{\prime}$,所以事实上,第一项是电单级项,即等效为所有电荷集中在一个点所构成的点电荷产生的场,第二项是偶极项
\begin{equation}
\begin{aligned}
  V_{\text{偶}}(\vb*{r})&=\frac{1}{4 \pi \epsilon_0}\frac{1}{r^2} \int r^{\prime} \cos \alpha \rho\left(\vb*{r}^{\prime}\right) d \tau^{\prime}\\
  &=\frac{1}{4\pi\epsilon_0}\frac{1}{r^2}\int \hat{\vb*{r}}\cdot \vb*{r'}\rho\left(\vb*{r}^{\prime}\right) d \tau^{\prime}\\
\end{aligned}
\end{equation}
事实上,$\hat{\vb*{r}}$与$\vb*{r'}$无关,可以提到积分外面去,而$\int \vb*{r'}\rho\left(\vb*{r}^{\prime}\right) d \tau^{\prime}=\vb*{p}$是整个构型的电偶极矩,所以
\begin{equation}
  V(\vb*{r})_{\text{偶}}=\frac{1}{4\pi\epsilon_0}\frac{\hat{\vb*{r}}\cdot \vb*{p}}{r^2}
\end{equation}
是电偶极子产生的电势,类似的,第三项是电四极子产生的电势...总之,多级展开告诉我们,在一个电荷构型的远处,电势等效为电荷量集中在一点的点电荷形成的电势(这体现了电荷值)再加上多级子的拟合(这体现了电荷分布的几何结构)。

\chapter{Magnetostatics}
\section{Lorentz Force Law}
洛伦兹力指出,对于一个在磁场中以速度$\vb*{v}$运动的带电量为$q$的电荷,其会受到磁场力
\begin{equation}
  \vb*{F}_{mag}=q\vb*{v}\times \vb*{B}
\end{equation}
当然,对于连续分布的电荷,有
\begin{equation}
\begin{aligned}
	 \vb*{ F}_{mag}&=\int \vb*{v}\times \vb*{B} \dd q\\
	 &=\int \lambda \vb*{v}\times \vb*{B} \dd l
\end{aligned}
\end{equation}
实际上
\begin{equation}
  I\vb*{\hat l}=\dv{q\hat{\vb*{l}}}{t}=\frac{\lambda\dd \vb*{l}}{\dd t}=\lambda \vb*{v}
\end{equation}
$\vb*{l}$为电流流经的方向,所以
\begin{equation}
  \vb*{F}_{mag}=\int I\dd \vb*{l}\times \vb*{B}
\end{equation}
这也被称为安培力。
在这里,顺便讨论几种电流密度:首先是面电流密度,面电荷密度的物理图像如下,其描绘的是在二维平面上流过的电流在垂直于流动方向的密度
\begin{figure}[H]
\centering
  \includegraphics[width=5cm]{K.png}
  \caption{面电荷密度}
\end{figure}

\begin{equation}
  \vb*{K}=\dv{I\hat{\vb*{l_\parallel}}}{l_\perp}=\frac{\sigma \dd S\hat{\vb*{l_\parallel}}}{\dd l_\perp\dd t}=\frac{\sigma\dd \vb*{l_{\parallel}}\dd l_\perp}{\dd l_\perp\dd t}=\sigma \vb*{v}
\end{equation}
体电荷密度:
\begin{figure}[H]
\centering
  \includegraphics[width=4cm]{J.png}
  \caption{体电荷密度}
\end{figure}
\begin{equation}
  \vb*{J}=\dv{I\hat{\vb*{l_\parallel}}}{S_\perp}=\frac{\rho \dd S_\perp \dd \vb*{l_\parallel}}{\dd S_\perp\dd t}=\rho \vb*{v}
\end{equation}
电流满足连续性方程:
\begin{equation}
  \iiint \div{\vb*{J}} \dd V =\oiint \vb*{J}\cdot \dd \vb*{S}
\end{equation}
而电流体密度的通量就是这个面包围的电流的净流出量,所以
\begin{equation}
  \oiint \vb*{J}\cdot \dd \vb*{S}=-I_{\text{净流出}}=-\dv{q}{t}=-\iiint \pdv{\rho}{t} \dd V
\end{equation}
结合(3.7),(3.8),可以得到电流的连续性方程
\begin{equation}
  \div{\vb*{J}}=-\pdv{\rho}{t}
\end{equation}
对于恒稳电流,其电流强度不随时间变化,则相应的,电流密度与电荷密度也不随时间变化,这就导致
\begin{equation}
	\div{\vb*{J}}=-\pdv{\rho}{t}=0
\end{equation}


\section{Biot-Savart Law}
比奥-萨法尔定律是针对于恒稳电流的,其给出:一个恒稳电流元,会在空间中产生磁场,大小为
\begin{equation}
  \dd \vb*{B} =\frac{\mu_0}{4\pi}\frac{I\dd \vb*{l}\times \hr}{\rr^2}
\end{equation}
同时可以给出他的积分形式
\begin{equation}
  \vb*{B}=\frac{\mu_0}{4\pi}I\int\frac{\dd \vb*{l}\times \hr}{\rr^2}=\frac{\mu_0}{4\pi}\iint \frac{\vb*{K}\times \hr}{\rr^2}\dd S=\frac{\mu_0}{4\pi}\iiint\frac{\vb*{J}\times \hr}{\rr^2}\dd V
\end{equation}

\section{Properties of Magnetic Field}
\subsection{Divergence}
磁场的散度可以通过对$(3.12)$求散度并利用矢量微分公式(推导可以见我的四元数讲义,那里给出了一个较优雅的推导方法)
\begin{equation}
\nabla \cdot\left(\vb*{J} \times \frac{\hat{\br}}{\rr^2}\right)=\frac{\hat{\br}}{\rr^2} \cdot(\nabla \times \vb*{J})-\vb*{J} \cdot\underbrace{\left(\nabla \times \frac{\hat{\br}}{\rr^2}\right)}_{=0}
\end{equation}
而由于$\nabla$是对场点求微分,但$\vb*{J}$仅是源点的函数,所以$\curl{\vb*{J}}=0$,于是
\begin{equation}
  \nabla\cdot \vb*{B}=0
\end{equation}

\subsection{Curl}
磁场的旋度同样也是对比奥-萨法尔定律求旋度得到的:
\begin{equation}
  \curl{\vb*{B}}=\mu_0 \vb*{J}
\end{equation}
其积分形式:
\begin{equation}
  \oint \vb*{B}\cdot \dd \vb*{l} =\mu_0 I_{\text{in}}
\end{equation}

\section{The vb*tor Potential of Magnetic Field}
与根据静电场的散度为0而引入标量势相似,对于静磁场,其旋度为0,因此,我们可以引入矢量势。

由于$\div{\vb*{B}}=0$,而根据矢量分析,旋度场是无源的,即$\div{(\curl{\vb*{A}}})=0$,所以可以引入
\begin{equation}
  \vb*{B}=\curl{\vb*{A}}
\end{equation}
$\vb*{A}$被称为静磁场的矢势。与静电场一样,$\vb*{A}$的选择也是有任意性的,可以在$\vb*{A}$的基础上加上任意的无旋量,得到的场强不变。但我们可以选择一种简便的取法,来让问题得到简化。

首先考虑磁场的旋度:
\begin{equation}
  \curl{\vb*{B}}=\mu_0 \vb*{J}\Rightarrow \curl{(\curl{\vb*{A}})}=\mu_0 \vb*{J}
\end{equation}
而
\begin{equation}
  \curl{(\curl{\vb*{A}})}=\vb*{\nabla}(\div{\vb*{A}})-\nabla^2 \vb*{A}
\end{equation}
一个简化的方法是令$\div{\vb*{A}}=0$(Coulomb规范),此时
\begin{equation}
  \curl{(\curl{\vb*{A}})}=-\nabla^2 \vb*{A}\Rightarrow \nabla^2 \vb*{A}=-\mu_0\vb*{J}
\end{equation}
这个形式仍然是Poisson方程,可以推断在Coulomb规范下的矢势形式为
\begin{equation}
  \vb*{A}=\frac{\mu_0}{4\pi}\iiint \frac{\vb*{J}}{\rr}\dd V'
\end{equation}
同样的,可以类似(3.12),来给出其他形式的电流分布情况的解。
\section{Boundary Condition}
通过$\vb*{B}$的散度与旋度,可以得到磁场的边界条件
\begin{align}
	\begin{cases}
		\div{\vb*{B}}=0\\
		\curl{\vb*{B}}=\mu_0 \vb*{J}
	\end{cases}
	\Rightarrow
	\begin{cases}
		\Delta \vb*{B}^\perp=0\\
		\Delta \vb*{B}^\parallel =\mu_0 \vb*{K}
	\end{cases}
	\quad \text{or}\quad \Delta \vb*{B}=\mu_0 \vb*{K}\times \vb*{\hat{n}}
\end{align}
证明方式与电场的边界条件一样。

也可以求出$\vb*{A}$的边界条件
\begin{align}
	\begin{cases}
		\div{\vb*{A}}=0\\
		\curl{\vb*{A}}=\vb*{B}
	\end{cases}
	\Rightarrow
	\begin{cases}
		\Delta \vb*{A}^\perp=0\\
		\Delta \vb*{A}^\parallel =0(\text{无穷小积分为0})
	\end{cases}
	\quad \text{or}\quad \Delta \vb*{A}=0
\end{align}
\section{Magnetic Field in Matter}
下面来研究在介质中的磁场。在介质中,根据介质对外磁场的响应可以将介质分为:顺磁性,抗磁性,铁磁性。
\subsection{Paramagnetism}
顺磁性是指,在外磁场作用下,介质被磁化的结果是加强外磁场。顺磁性的来源是原子的磁矩(这个过程事实上自旋和轨道磁矩都参与了作用,我个人认为Griffiths的书上把顺磁性归结于自旋磁矩而把抗磁性归结于轨道磁矩有一定的误导性,实际上自旋和轨道磁矩都有作用),在外加磁场下,磁矩方向倾向于向着磁场方向排列(可以想像一磁偶极子,即小环形电流,在外磁场下会倾向于指向外磁场的方向)。
\begin{figure}[H]
\centering
  \includegraphics[width=5cm]{paramag.png}
  \caption{顺磁性}
\end{figure}
\subsection{Diamagnetism}
外磁场对电子轨道的作用形成了抗磁性。
\begin{figure}[H]
\centering
  \includegraphics[width=5cm]{diamag.png}
  \caption{抗磁性}
\end{figure}
如图,在不加外磁场时,电子由于库伦作用绕轨道运动,遵循
\begin{equation}
  \frac{1}{4\pi\epsilon_0}\frac{e^2}{R^2}=m\frac{v^2}{R}
\end{equation}
其会产生一轨道磁矩
\begin{equation}
  \vb*{m}=-IS\hat{\vb*{z}}=-\frac{ev}{2\pi R}\pi R^2\hat{\vb*{z}}=-\frac{1}{2}evR\hat{\vb*{z}}
\end{equation}

假设电子轨道半径是固定的,若在$-\hat{\vb*{z}}$方向加一磁场$\vb*{B}$(这里之所以与书上相反是因为考虑到轨道磁矩在外磁场的变化下会赚到与外磁场同向),电子还会受到洛伦兹力的作用,此时
\begin{equation}
  \frac{1}{4\pi\epsilon_0}\frac{e^2}{R^2}-e\tilde{v}B=m\frac{\tilde{v}^2}{R}
\end{equation}
比较可得$\tilde{v}<v$,也就是在磁场作用下,电子运动减速。由(3.25),$\abs{\vb*{\tilde m}}<\abs{\vb*{m}}$,轨道磁矩减小,即磁矩相当于有一个向上的变化,这将产生一个向上的磁场,这个磁场与外磁场叠加就会削弱外磁场(因为方向相反),产生抗磁性。
\subsection{Conclusion of Magnetism}
根据上面的讨论,我们不难发现,抗磁性实际上是各种原子都有的,因为无论什么方向的轨道磁矩,在外场的作用下都有一定的削弱作用,但抗磁矩是较为微弱的,顺磁矩的强度会强一些。所以只有在\textbf{满壳层}的原子中,我们才能观察到抗磁矩。(这种情况下,自旋与轨道角动量均为0,即$\vb*{L}=\vb*{S}=0$)
\subsection{Potential}
我们已经知道了介质在磁场中会被磁化,并且明白了磁化的一个经典解释,现在的问题是,如果给定了一个磁化强度$\vb*{M}=\frac{\sum \vb*{m}}{\dd \tau}$,他会产生一个什么样的势呢?

一个磁偶极子的矢势为
\begin{equation}
  \vb*{A}(\vb*{r})=\frac{\mu_0}{4\pi}\frac{\vb*{m(r')}\times\hr}{\rr^2}
\end{equation}
和处理电极化问题一样,在这里,为了得到总体的矢势,要对其做积分
\begin{equation}
\begin{aligned}
	   \vb*{A}(\vb*{r})&=\frac{\mu_0}{4\pi}\iiint \frac{\vb*{M(r')}\times\hr}{\rr^2}\dd \tau'\\
	   &=\frac{\mu_0}{4\pi}\iiint \vb*{M(r')}\times\nabla'\frac{1}{\rr}\dd \tau'\\
	   &=\frac{\mu_0}{4\pi}\left( \iiint \frac{\curl{\vb*{M}}}{\rr}\dd \tau' +\oiint \frac{\vb*{M}}{\rr}\times \dd \vb*{S} \right)\\
	   &=\frac{\mu_0}{4\pi}\left( \iiint \frac{\curl{\vb*{M}}}{\rr}\dd \tau' +\oiint \frac{\vb*{M}}{\rr}\times \vb*{\hat{n}}\dd S \right)
\end{aligned}
\end{equation}
与电介质一样,该式第一项看起来像体束缚电流密度对矢量势的贡献,而第二项看起来像表面的面束缚电流密度的贡献,所以
\begin{equation}
  \curl{\vb*{M}}=\vb*{J_b}\quad,\quad \vb*{M}\times \vb*{\hat n}=\vb*{K_b}
\end{equation}
\subsection{The Auxiliary Field}
在介质中安培定律需要改写为
\begin{equation}
  \curl{\vb*{B}}=\mu_0(\vb*{J_b+J_f})
\end{equation}
而$(3.29)$给出了$\vb*{J_b}$的形式,代入有
\begin{equation}
  \curl{\left(\frac{\vb*{B}}{\mu_0}-\vb*{M}\right)}=\vb*{J_f}
\end{equation}
可以定义一个辅助场
\begin{equation}
  \vb*{H}=\frac{\vb*{B}}{\mu_0}-\vb*{M}
\end{equation}
这会将介质中的安培定律简化为
\begin{equation}
  \curl{\vb*{H}}=\vb*{J_f}
\end{equation}
\subsection{Linear Media}
在线性介质中,磁化率可以写作辅助场的正比
\begin{equation}
  \vb*{M}=\chi_m \vb*{H}
\end{equation}
代入(3.32)
\begin{equation}
  \vb*{B}=\mu_0(1+\chi_m)\vb*{H}
\end{equation}
可以定义$1+\chi_m=\mu_r$,这样在线性介质中
\begin{equation}
  \vb*{B}=\mu_0\mu_r \vb*{H}
\end{equation}
\subsection{Boundary Condition}
在介质中,场的散度和旋度为
\begin{align}
	\begin{cases}
		\div{\vb*{B}}=0\\
		\curl{\vb*{H}}=\vb*{J_f}
	\end{cases}
\end{align}
同样的,在边界做闭合曲面积分和环路积分,可以得到介质中的边界条件
\begin{align}
	\begin{cases}
		\Delta \vb*{H}^\perp =-\Delta \vb*{M}^\perp\\
		\Delta \vb*{H}^\parallel =\vb*{K_f}\times \hat{\vb*{n}}
	\end{cases}
\end{align}

\part{Electrodynamics}
\chapter{Electrodynamics}
\section{Ohm's Law and Electromotive Force}
我们已经知道,在电流周围会产生磁场,那么接下来的问题是:什么产生了电流?要知道,电流是电子受力流动的结果,所以一个简单的近似就是令电流密度
\begin{equation}
  \vb*{J}=\sigma \vb*{f}
\end{equation}
其中比例系数$\sigma$被定义为电导率,$\vb*{f}$是单位电荷的受力,在这里,我们考虑电荷受电磁力的作用
\begin{equation}
  \vb*{J}=\sigma(\vb*{E}+\vb*{v}\times \vb*{B})
\end{equation}
由于电子的有效速度$\vb*{v}$一般较小,可以不考虑洛伦兹力的作用,则
\begin{equation}
  \vb*{J}=\sigma \vb*{E}
\end{equation}
这就是欧姆定律,对于金属等电导率很大($\sim 10^8$)的物体,只需要很小的电场就可以得到有限的电流啦。欧姆定律的一个比较耳熟能详的形式是宏观形式:
\begin{equation}
  J=\frac{I}{S}=\sigma E=\sigma \frac{U}{l}\Rightarrow U=I\frac{l}{S\sigma}=I\frac{\rho l}{S}=IR
\end{equation}
所以欧姆定律告诉我们,在导线中的电流,一般是电场力驱动的,但可以想象,在这个电场的作用下,正负电荷很快就会累积在这个电场的两端来抵消这个电场,而在恒稳电流中什么维持了这个电场呢?

很显然,依靠静电力是做不到的,对于静电场,有$\oint \vb*{E}\cdot \dd \vb*{l}=0$所以一个电子在静电场中环绕一圈时,静电场做的功为0,如果存在用电器,电场的能量会很快的被消耗掉,这个电路是稳定不了的,所以形成电流是需要依靠电源中的非静电力$f_s$(在电源中,同时存在$\vb*{f}=\vb*{f_s}+\vb*{E}$),实际上在这里非静电力不是力的量纲,而是单位电荷感受到的非静电力,是场的量纲。可以定义电动势
\begin{equation}
  \epsilon=\oint \vb*{f} \cdot\dd \vb*{l}=\oint \vb*{f_s}\cdot\dd \vb*{l}=\sum_i \int \vb*{f_s}\cdot \dd \vb*{l_i}(\text{对每个电源求和})
\end{equation}
所以对于某个理想电源内,
\begin{equation}
  \epsilon=\int \vb*{f_s}\cdot\dd \vb*{l}=-\int \vb*{E}\cdot \dd \vb*{l}=V
\end{equation}
我们可以看到,实际上,非静电力为电源两端提供了一个电势差,这个稳定的电势差形成了一个匀强电场,来形成电流。
\section{Faraday's Law}
法拉第通过实验总结出:穿过一个回路中的磁通量变化时,会产生电动势,其大小为:
\begin{equation}
  \epsilon=-\pdv{\phi}{t}=-\pdv{}{t}\iint\vb*{B}\cdot \dd \vb*{S}
\end{equation}
根据磁通量的变化方式,可以分为动生电动势和感生电动势。动生电动势是指穿过回路的磁场强度不变,但穿过回路的面积发生了变化,此时
\begin{equation}
  \epsilon\xlongequal[]{\text{匀强磁场}}-\vb*{B}\cdot\pdv{\vb*{S}}{t}\xlongequal[]{\text{方形回路}}Blv
\end{equation}
感生电动势是指,磁场强度发生了变化而面积不变,此时
\begin{equation}
  \epsilon=-\iint\pdv{\vb*{B}}{t}\cdot\dd \vb*{S} 
\end{equation}
法拉第认为,变化的磁场实际上是通过产生感生电场作为非静电力来产生的感生电动势,即
\begin{equation}
  \epsilon=-\iint\pdv{\vb*{B}}{t}\cdot\dd \vb*{S} =\oint \vb*{E}\cdot\dd \vb*{l}
\end{equation}
上式被称为法拉第定律,由此可以看到,感生电场的回路积分不为0,感生电场是有旋无源场。应用Stokes定律,可以得到$(4.10)$的微分形式:
\begin{equation}
  \curl{\vb*{E}}=-\pdv{\vb*{B}}{t}
\end{equation}
\section{Maxwell Equations}
\subsection{Maxwell Equations in Vacuum}
现在,我们可以整理一下在目前为止电磁场中的四个微分方程:
\begin{align}
        \begin{cases}
                \div{\vb*{E}}=\frac{\rho}{\epsilon_0}\\
                \curl{\vb*{E}}=-\pdv{\vb*{B}}{t}\\
                \div{\vb*{B}}=0\\
                \curl{\vb*{B}}=\mu_0\vb*{J}
        \end{cases}
\end{align}
但你可能会发现,这个方程存在不自洽!对第四式两边同时求散度,有
\begin{equation}
  \div{(\curl{\vb*{B}})}=\mu_0 \div{\J}
\end{equation}
注意到旋度场是无源的,所以上式左边为0,这就要求
\begin{equation}
  \mu_0 \div{\J}=0
\end{equation}
但Maxwell方程组推导的一个必要条件是电流连续性方程
\begin{equation}
  \div{\J}=-\pdv{\rho}{t}
\end{equation}
在含时情况下,一般不要求$\pdv{\rho}{t}$为0,所以这个方程组并不自洽.
(4.13)中左边一定为0,所以为了让右边也为0,需要在(4.12)第四式右边增加一个量,使其散度为$\mu_0\pdv{\rho}{t}$.

这是容易的,注意到Maxwell方程组的方程一右边有$\rho$,将其对t求导
\begin{equation}
  \div{\pdv{\vb*{E}}{t}}=\frac{1}{\epsilon_0}\pdv{\rho}{t}
\end{equation}
于是可以令第四式改写为
\begin{equation}
  \curl{\vb*{B}}=\mu_0\vb*{J}+\mu_0\epsilon_0\pdv{E}{t}
\end{equation}
可以看出,$\epsilon_0\pdv{E}{t}$有电流密度的量纲,称为位移电流密度。事实上,这也指出变化的电场是可以产生磁场的

至此,Maxwell方程组完全自洽,1862年Maxwell提出了以他名字命名的方程组
\begin{align}
        \begin{cases}
                \div{\vb*{E}}=\frac{\rho}{\epsilon_0} & \text{Gauss's law}\\
                \curl{\vb*{E}}=-\pdv{\vb*{B}}{t} & \text{Faraday's law}\\
                \div{\vb*{B}}=0\\
                \curl{\vb*{B}}=\mu_0\vb*{J}+\mu_0\epsilon_0\pdv{\vb*{E}}{t}& \text{Ampere's law with Maxwell's correction}
        \end{cases}
\end{align}
即真空中的Maxwell方程组。

\subsection{Maxwell Equations in Matter}
下面来考虑介质中的情况。

在静场中,我们已经推出了$-\div{\vb*{P}}=\rho_b,\curl{\vb*{M}}=\vb*{J_b}$,但在动场情况下,电极化强度$\vb*{P}$可能会随时间变化,这会引起电荷密度的变化而引起电流$\vb*{J_p}$。
\begin{figure}[H]
\centering
  \includegraphics[width=5cm]{Jp.png}
\end{figure}
如上图,考虑一个简单的情况,由(1.30)可得$\sigma_b=P$,此时若$P=P(t)$,则$\sigma_b=\sigma_b(t)$,在这个小区域内会产生电流
\begin{equation}
  \dd I=\pdv{\sigma}{t}\dd a_\perp=\pdv{P}{t}\dd a_\perp
\end{equation}
所以变化的电极化强度会在自由电流和束缚电流之外,形成$\vb*{J_p}$:
\begin{equation}
  \vb*{J_p}=\pdv{\vb*{P}}{t}
\end{equation}
求散度:
\begin{equation}
  \div{\vb*{J_p}}=\pdv{}{t}\div{\vb*{P}}=-\pdv{\rho_b}{t}
\end{equation}
所以在动场情况下,介质中存在两种电荷:自由电荷、束缚电荷;三种电流:自由电流、束缚电流与$\vb*{J_p}$。
考虑Gauss定理:
\begin{equation}
  \div{\vb*{E}}=\frac{\rho_f+\rho_b}{\epsilon_0}
\end{equation}
仍然有
\begin{equation}
  \div{\vb*{D}}=\rho_f\quad(\vb*{D}=\epsilon_0 \vb*{E} + \vb*{P})
\end{equation}
Ampere定理:
\begin{equation}
  \curl{\vb*{B}}=\mu_0(\J_f+\J_b+\J_p+\epsilon_0\pdv{\vb*{E}}{t})=\mu_0(\J_f+\curl{\vb*{M}}+\pdv{\vb*{(\epsilon_0 \vb*{E}+P)}}{t})
\end{equation}
所以
\begin{equation}
  \curl{\left(\frac{\vb*{B}}{\mu_0}-\vb*{M}\right)}=\curl{\vb*{H}}=\vb*{J_f}+\pdv{\vb*{D}}{t}
\end{equation}
由此,我们可以整理出介质中的Maxwell方程组:
\begin{align}
	\begin{cases}
		\div{\vb*{D}}=\rho_f\\
		\curl{\vb*{E}}=-\pdv{\vb*{B}}{t}\\
		\div{\vb*{B}}=0\\
		\curl{\vb*{H}}= \vb*{J_f}+\pdv{\vb*{D}}{t}
	\end{cases}
\end{align}

\section{Boundary Condition}
又到了一章的最后,按照“惯例”,我们来给出边界条件:

\begin{align}
	\begin{cases}
		\div{\vb*{D}}=\rho_f\\
		\curl{\vb*{E}}=-\pdv{\vb*{B}}{t}\\
		\div{\vb*{B}}=0\\
		\curl{\vb*{H}}= \vb*{J_f}+\pdv{\vb*{D}}{t}
	\end{cases}\Rightarrow
	\begin{cases}
		\Delta \vb*{D}^\perp =\sigma_f\\
		\Delta \vb*{E}^\parallel = 0\\
		\Delta \vb*{B}^\perp = 0\\
		\Delta \vb*{H}^\parallel =\vb*{K_f}\times \vb*{\hat n}
	\end{cases}
\end{align}
而对于线性介质,在没有自由电荷和自由电流分布的区域,边界条件会变得简单
\begin{align}
	\begin{cases}
		\epsilon_1\vb*{E_1}^\perp=\epsilon_2\vb*{E_2}^\perp\\
		\vb*{E_1}^\parallel=\vb*{E_2}^\parallel\\
		\vb*{B_1}^\perp=\vb*{B_2}^\perp\\
		\frac{\vb*{B_1}^\parallel}{\mu_1}=\frac{\vb*{B_2}^\parallel}{\mu_2}
	\end{cases}
\end{align}


\chapter{Conservation Laws}
在这一章里,我们来考虑电动力学中的守恒律:电荷守恒、能量守恒、动量守恒、角动量守恒。
\section{Conservation of Charge}
广义的电荷守恒是指在整个宇宙范围内,电荷不会增加也不会减少,他将保持为一个不变量,而局域的电荷守恒是指在某一区域内电荷的增加或减少是电荷流入这个区域或从这个区域流出造成的,即电荷不会凭空的产生或湮灭,这其实就是电荷的连续性方程:
\begin{equation}
  \div{\vb*{J}}=-\pdv{\rho}{t}
\end{equation}
\section{Conservation of Energy}
电磁场的能量也应该是守恒的,为此,我们考虑电磁场对一个电荷的做功:
\begin{equation}
  \dd W =q\vb*{E}\cdot \dd \vb*{l}=q\vb*{E}\cdot \vb*{v}\dd t
\end{equation}
所以电磁场做功的功率为:
\begin{equation}
  \dv{W}{t}=q \vb*{E}\cdot\vb*{v}=\iiint \rho \vb*{E}\cdot\vb*{v}\dd \tau=\iiint \vb*{E}\cdot\vb*{J}\dd \tau
\end{equation}
根据Maxwell方程组,
\begin{equation}
\vb*{J}= \frac{1}{\mu_0} \curl{\vb*{B}}-\epsilon_0\pdv{\vb*{E}}{t}
\end{equation}
\begin{equation}
\begin{aligned}
	  \vb*{E}\cdot \vb*{J}&=\frac{1}{\mu_0}\vb*{E}\cdot\curl{\vb*{B}}-\epsilon_0\vb*{E}\cdot\pdv{\vb*{E}}{t}\\
	  &=\frac{1}{\mu_0}(-\div{(\vb*{E}\times \vb*{B})}+\vb*{B}\cdot\curl{\vb*{E}})-\frac{\epsilon_0}{2}\pdv{E^2}{t}\\
	  &=\frac{1}{\mu_0}(-\div{(\vb*{E}\times \vb*{B})}-\vb*{B}\cdot\pdv{\vb*{B}}{t})-\frac{\epsilon_0}{2}\pdv{E^2}{t}\\
	  &=-\frac{1}{\mu_0}\div{(\vb*{E}\times \vb*{B})}-\pdv{\omega_{em}}{t}
\end{aligned}
\end{equation}
其中,$\omega_{em}=\frac{\epsilon_0 E^2}{2}+\frac{B^2}{2\mu_0}$是电磁场的能量密度,于是
\begin{equation}
  \dv{W}{t}=-\frac{1}{\mu_0}\oiint (\vb*{E}\times \vb*{B})\cdot\dd \vb*{a}-\iiint \pdv{\omega_{em}}{t}\dd \tau
\end{equation}
定义玻印廷矢量
\begin{equation}
  \vb*{S}=\frac{1}{\mu_0}\vb*{E}\times \vb*{B}
\end{equation}
于是
\begin{equation}
    \dv{W}{t}=-\oiint \vb*{S}\cdot\dd \vb*{a}-\iiint \pdv{\omega_{em}}{t}\dd \tau
\end{equation}
(5.6)或(5.8)也被称为玻印廷定理,这表示电磁场对电荷的做功体现在其空间内部剩余电磁场能量的减少以及能量从边界上的流失,由于这个原因,也可以叫$\vb*{S}$为电磁场的能流密度。

当然,电磁场对电荷做功会改变电荷的机械能,使用$\omega_{mesh}$表示机械能密度,(5.8)还可以写作
\begin{equation}
  \pdv{}{t}\iiint (\omega_{mesh}+\omega_{em}) \dd \tau=-\oiint \vb*{S}\cdot\dd \vb*{a}
\end{equation}
应用Gauss公式,可以得到微分形式:
\begin{equation}
  \div{\vb*{S}}=-\pdv{}{t} \left(\omega_{mesh}+\omega_{em}\right)
\end{equation}
这与电荷连续性方程有类似的形式.
\section{Conservation of Momentum}
电磁场不仅携带能量,还可以携带动量,电磁场的动量密度为
\begin{equation}
  \vb*{\mathcal{P}_{em}}=\epsilon_0 \vb*{E}\times \vb*{B}
\end{equation}
动量守恒:
\begin{equation}
  \pdv{}{t} \left(\mathcal{P}_{em}+\mathcal{P}_{mesh}\right)=\div{\overleftrightarrow{\boldsymbol{T}}}
\end{equation}
$\overleftrightarrow{\boldsymbol{T}}$是麦克斯韦应力张量,而$-\overleftrightarrow{\boldsymbol{T}}$是动量流密度,这样(5.12)就与电荷守恒、能量守恒有着相同的形式了。
\section{Angular Momentum}
由于在经典力学中,$\vb*{l}=\vb*{r}\times \vb*{p}$,所以电磁场的角动量密度
\begin{equation}
  \vb*{\mathcal{L}}_{em}=\vb*{r}\times \vb*{\mathcal{P}}_{em}=\epsilon_0 \vb*{r}\times(\vb*{E}\times \vb*{B})
\end{equation}

\chapter{Electromagnetic Waves}
\section{Mechanical Waves}
\subsection{The Wave Equation}
波动是指一个扰动在介质中以某一速度的传播,其方程满足
\begin{equation}
  f(z,t)=f(z-vt,0)=g(z-vt)
\end{equation}
其中$v$是扰动在介质中传输的速度。为了得到一般的波动方程,我们来研究扰动在一均匀弦上的传播。将弦的两端固定,在某一处将其拉起而偏离平衡位置,如下图
\begin{figure}[H]
\centering
  \includegraphics[width=5cm]{string.png}
  \caption{弦的扰动}
\end{figure}
$f$表示其偏离平衡位置的位移。假定其$f$很小,且弦只在上下振动而不发生左右平移,则对于这一小段弦,根据牛顿第二定律,有
\begin{equation}
  \begin{aligned}
  	F&=ma\\
  	T(\sin{\theta'}-\sin\theta)&=\mu \Delta z\pdv[2]{f}{t}\\
  	T(\pdv{f}{z}_{z+\Delta z}-\pdv{f}{z}_{z})/{\Delta z}&=\mu\pdv[2]{f}{t}\\
  	\pdv[2]{f}{z}&=\frac{\mu}{T}\pdv[2]{f}{t}\xlongequal[]{v=\sqrt{\frac{T}{\mu}}}\frac{1}{v^2}\pdv[2]{f}{t}
  \end{aligned}
  \end{equation}
这就是一般的波动方程,其满足性质:
\begin{equation}
  f(z,t)=g(z-vt)
\end{equation}
所以在这个意义上
\begin{equation}
  v=\frac{T}{\mu}
\end{equation}
可以被理解为波在这一介质中传播的速度。
\subsection{Sinusoidal Waves}
在有周期性的波中,正弦波可以说是最简单的一类波,任何周期波都可以用正弦波来展开,所以研究正弦波的一些性质是有意义的。
\begin{figure}[H]
\centering
  \includegraphics[width=10cm]{sin.png}
  \caption{正弦波图像}
\end{figure}
正弦波(当然这里是一个余弦函数,不过也没什么问题)可以定义为
\begin{equation}
f(z, t)=A \cos [k(z-v t)+\delta]
\end{equation}
其中$A$为振幅、$k=\frac{2\pi}{\lambda}$为波数,体现其空间周期性、$\delta$为初相位、$v$为波速,所以
\begin{equation}
  kv=\frac{2\pi}{\lambda}v=\frac{2\pi}{T}=\omega
\end{equation}
体现其时间周期性。
所以你也可以把正弦波写成
\begin{equation}
  f(z, t)=A \cos (kz-\omega t+\delta)
\end{equation}
以上形式描述了一个向z的正半轴传播的正弦波,如果你要研究的波是有同样样式的、向z轴负半轴传播的波,你需要在时间因子(这意味着$f=g(z+vt)$)和初相位(这意味着初相位体现在了z的正半轴)上加一个负号
\begin{equation}
    f(z, t)=A \cos (kz+\omega t-\delta)
\end{equation}
\begin{figure}[H]
\centering
  \includegraphics[width=6cm]{sin2.png}
  \caption{反向传播的正弦波}
\end{figure}
\!\!\!\!\!\!\!\!\!根据余弦函数的性质,也可以改写为
\begin{equation}
     f(z, t)=A \cos (-kz-\omega t+\delta)
\end{equation}
与式$(6.7)$对比,为了得到一个传输方向相反,其他特性不变的波,你只需要在空间因子上加一个负号(这在数学形式上变得简便了一些,但是并没有物理意义)。

当然,除了常规的三角函数表示法,利用欧拉公式$e^{\mathrm{i}\theta}=\cos \theta+\i\sin\theta $,还可以将正弦波函数改写为指数形式
\begin{equation}
  f(z,t)=Re[Ae^{\i(kz-\omega t+\delta)}]=Re[Ae^{\i \delta}e^{\i(kz-\omega t)}]=Re[\tilde Ae^{\i(kz-\omega t)}]=Re[\tilde f(z,t)].
\end{equation}


\subsection{Reflection and Transmission}
有了波函数(其实我不知道应不应该叫他波函数,波函数貌似是量子力学中的术语,如果有知道的朋友可以告诉我),我们可以来研究波动在不同介质中的传播性质。

考虑一个波从介质1传播到介质2,在介质12中,波的传播速度不一样,分别为$v_1,v_2$,但周期T是一样的(假定震源处以恒定频率进行扰动)所以在两个介质中波长
\begin{equation}
  \lambda_i=v_iT
\end{equation}
是不一样的,相应的,波数k也不一样,假定其分别为$k_1,k_2$于是波在两个介质的分界面处可能发生反射或透射,入射、反射、透射波分别可以写作
\begin{equation}
  \begin{aligned}
  \begin{cases}
  		  f_{\mathrm{Incide}}(z,t)&=A e^{\i ( k_1z-\omega t) }\\
  		  f_{\mathrm{Reflect}}(z,t)&= R e^{\i (-k_1 z - \omega t)}\\
  		  f_{\mathrm{Transmiss}}(z,t)&= T e^{\i (k_2 z - \omega t)}
  \end{cases}
\end{aligned}
\end{equation}
在写下这三个式子时,其并没有包含什么性质,因为这只不过体现了波动的一般形式,所以为了给出$R,T$与A的关系,我们还需要给出边界条件。对于这个问题,在边界上波函数f应当满足的条件应该是连续,而根据式(6.2)和图6.1,在边界上,如果没有有限质量的节点,则波函数的导数也应该是连续的,即

  \begin{align}
	  f_I(0,t)+f_R(0,t)=f_T(0,t)	\\
	  f'_I(0,t)+f'_R(0,t)=f'_T(0,t)
\end{align}
(这里的$'$是对z求导)
这将会得到
\begin{align}
\begin{cases}
	A+R=T\\
	Ak_1-Rk_1=Tk_2
\end{cases}
\end{align}
当然,$v$是可测量量,可以使用k和v的反比关系将其改写为
\begin{align}
\begin{cases}
	A+R=T\\
	Av_2-Rv_2=Tv_1
\end{cases}
\end{align}
解得
\begin{align}
\begin{cases}
	R=\left(\frac{v_2-v_1}{v_2+v_1}\right) A\\
	T=\left(\frac{2 v_2}{v_2+v_1}\right) A\\
\end{cases}
\end{align}


\section{Electromagnetic Waves in Vacuum}
在这一节中,我们来研究电磁波在真空中的性质。
\subsection{Derivation of Wave Equation}
在真空中的电磁性质由真空中的Maxwell方程组给出:
\begin{align}
        \begin{cases}
                \div{\vb*{E}}=\frac{\rho}{\epsilon_0}\\
                \curl{\vb*{E}}=-\pdv{\vb*{B}}{t} \\
                \div{\vb*{B}}=0\\
                \curl{\vb*{B}}=\mu_0\vb*{J}+\mu_0\epsilon_0\pdv{\vb*{E}}{t}& 
        \end{cases}
\end{align}
如果电磁波已经产生,并在无源区域传播,此时方程中的源可以去掉
\begin{align}
        \begin{cases}
                \div{\vb*{E}}=0\\
                \curl{\vb*{E}}=-\pdv{\vb*{B}}{t} \\
                \div{\vb*{B}}=0\\
                \curl{\vb*{B}}=\mu_0\epsilon_0\pdv{\vb*{E}}{t}& 
        \end{cases}
\end{align}
这个方程看起来比较对称,为了得到关于$\vb*{E}$或$\B$本身的函数,应进行退偶操作,即消去关于$\E$的方程中的$\B$与关于$\B$的方程中的$\E$,为此,对(6.19)的第2式求旋度:
\begin{equation}
  \curl{\curl{\E}}=\grad(\div{\E})-\nabla^2\E=-\nabla^2\E=-\mu_0\epsilon_0\pdv[2]{\E}{t}
\end{equation}
同样的,对(6.19)的式4求旋度有
\begin{equation}
  \nabla^2\B=\mu_0\epsilon_0\pdv[2]{\B}{t}
\end{equation}
整理有
\begin{align}
\begin{cases}
	\nabla^2\E=\mu_0\epsilon_0\pdv[2]{\E}{t}\\
	\nabla^2\B=\mu_0\epsilon_0\pdv[2]{\B}{t}
\end{cases}
\end{align}
与波动方程$(6.2)$对比,得电磁波在真空中传播的速度是
\begin{equation}
  v=\frac{1}{\sqrt{{\epsilon_0\mu_0}}}
\end{equation}
由于$\frac{1}{\sqrt{\epsilon_0\mu_0}}$的数值与光速一样,所以Maxwell提出假设:光也是电磁波
\subsection{Monochromatic Plane Waves}
同机械波的形式一样,在电磁波中,波动方程的解的最简单形式是单色平面波,即
\begin{align}
	    \vb*{E}=\vb*{E}_0e^{\i(kz-\omega t)}\\
  \B=\B_0e^{\i(kz-\omega t)}	\notag
\end{align}
可以将其带入Maxwell方程组来获得他的一些性质,首先是代入(6.19)的1和3,以$\div{\E}=0$为例,有
\begin{equation}
  E_{0,z}ike^{\i(kz-\omega t)}=0\Rightarrow E_{0,z}=0
\end{equation}
同样的,可以导出$B_{0,z}=0$,这意味着电磁场没有沿着传输方向的分量,表明电磁波是一种横波。然后代入(6.19)的2和4,以$\curl{\vb*{E}}=-\pdv{\vb*{B}}{t}$为例,
\begin{equation}
\begin{aligned}
	  \mdet{\vb*{i} &\vb*{j}& \vb*{k}\\ \pdv{}{x} &\pdv{}{y} &\pdv{}{z}\\E_{0,x}e^{\i kz}&E_{0,y}e^{\i kz}&0}e^{-\i\omega t}&=\vb*{B_0}e^{\i kz}\i\omega e^{-\i \omega t}\\
	  \Rightarrow
	  \begin{cases}
	  -\vb*{i}E_{0,y}\i ke^{\i kz}&=\vb*{i}B_{0,x}e^{\i kz}\i\omega \\
	  &\Downarrow\\
	  \qquad \quad \,\,E_{0,y}k&=-B_{0,x}\omega
	  \end{cases}\qquad &\mathrm{and}\qquad 
	  	  \begin{cases}
	  \quad\vb*{j}E_{0,x}\i ke^{\i kz}&=\vb*{j}B_{0,y}e^{\i kz}\i\omega \\
	  &\Downarrow\\
	  \qquad \quad \,\,E_{0,x}k&=B_{0,y}\omega
	  \end{cases}
\end{aligned}
\end{equation}
于是有
\begin{equation}
  \B_0=\frac{k}{\omega}\vb*{\hat z}\times \E_0=\frac{1}{c}\vb*{\hat z}\times \E_0
\end{equation}
这意味着在电磁波的传播过程中$\vb*{B}$和$\E$是垂直的,且磁场的振幅是电场振幅的$\frac{1}{c}$。

以上是一维的情况,也可以把其推广到三维:
\begin{align}
	\begin{cases}
		\E(\vb*{r},t)=\vb*{E_0}e^{\i(\vb*{k}\cdot \vb*{r}-\omega t)}={E_0}e^{\i(\vb*{k}\cdot \vb*{r}-\omega t)}\vb*{\hat n}\\
		\B(\vb*{r},t)=\frac{1}{c}\vb*{\hat k}\times \vb*{E}=\frac{1}{c}E_0e^{\i(\vb*{k}\cdot \vb*{r}-\omega t)}(\vb*{\hat k}\times \vb*{\hat n})
	\end{cases}
\end{align}

\section{Electromagnetic Waves in Matter}
在介质中的场点处,Maxwell方程组为
\begin{align}
        \begin{cases}
                \div{\vb*{E}}=0\\
                \curl{\vb*{E}}=-\pdv{\vb*{B}}{t} \\
                \div{\vb*{B}}=0\\
                \curl{\vb*{B}}=\mu\epsilon\pdv{\vb*{E}}{t}& 
        \end{cases}
\end{align}
这与真空中的形式是一致的,为了将我们刚刚导出的公式拓展到介质中,只需要做代换$\mu_0\rightarrow\mu\quad,\quad \epsilon_0\rightarrow\epsilon$,于是波动方程为
\begin{align}
\begin{cases}
	\nabla^2\E=\mu\epsilon\pdv[2]{\E}{t}\\
	\nabla^2\B=\mu\epsilon\pdv[2]{\B}{t}
\end{cases}
\end{align}
所以电磁波在介质中的波速为
\begin{equation}
  v=\frac{1}{\sqrt{\epsilon\mu}}=\frac{c}{n}
\end{equation}
$n$叫做折射率,关于折射理论,我们将在下一节看到,总之,光(电磁波)在介质中的速度会下降一个因子n。

\section{Reflection and Transmission}
下面我们来考虑电磁波从一个介质射向另一个介质会发生什么(怎么感觉这个句子怪怪的哈哈哈哈 不过我语文不好,见谅!)
\begin{figure}[H]
\centering
  \includegraphics[width=5cm]{EMWaveIncide.png}
  \caption{电磁波的反射与透射}
\end{figure}
首先写出入射、反射、透射波的形式解(考虑t=0时刻):
\begin{equation}
\begin{aligned}
	{\vb*{E}}_I(\vb*{r}, t)={\vb*{E}}_{0_I} e^{i\vb*{k}_I \cdot \vb*{r}}, \quad {\vb*{B}}_I(\vb*{r}, t)=\frac{1}{v_1}\left(\hat{\vb*{k}}_I \times {\vb*{E}}_I\right)\\
	{\vb*{E}}_R(\vb*{r}, t)={\vb*{E}}_{0_R} e^{i\vb*{k}_R \cdot \vb*{r}}, \quad {\vb*{B}}_R(\vb*{r}, t)=\frac{1}{v_1}\left(\hat{\vb*{k}}_R \times {\vb*{E}}_R\right)\\
{\vb*{E}}_T(\vb*{r}, t)={\vb*{E}}_{0_T} e^{i\vb*{k}_T \cdot \vb*{r}}, \quad {\vb*{B}}_T(\vb*{r}, t)=\frac{1}{v_2}\left(\hat{\vb*{k}}_T \times {\vb*{E}}_T\right)
\end{aligned}
\end{equation}
在边界上,有边界条件
\begin{align}
	\begin{cases}
		\epsilon_1\vb*{E_1}^\perp=\epsilon_2\vb*{E_2}^\perp\\
		\vb*{E_1}^\parallel=\vb*{E_2}^\parallel\\
		\vb*{B_1}^\perp=\vb*{B_2}^\perp\\
		\frac{\vb*{B_1}^\parallel}{\mu_1}=\frac{\vb*{B_2}^\parallel}{\mu_2}
	\end{cases}
\end{align}
以边界条件2为例,这需要
\begin{equation}
  E_{0_I}\cos\theta_I e^{i\vb*{k}_I \cdot \vb*{r}}+E_{0_R}\cos\theta_R e^{i\vb*{k}_R \cdot \vb*{r}}=E_{0_T}\cos\theta_T e^{i\vb*{k}_T \cdot \vb*{r}}
\end{equation}
要使得这个条件在整个$z=0$平面上成立,必须附加一个边界条件:在$z=0$时,指数因子相同,即
\begin{equation}
  \vb*{k}_I \cdot \vb*{r}=\vb*{k}_T \cdot \vb*{r}=\vb*{k}_R \cdot \vb*{r}
\end{equation}
或
\begin{equation}
  x(k_I)_x+y(k_I)_y=x(k_T)_x+y(k_T)_y=x(k_R)_x+y(k_R)_y
\end{equation}
在$x=0$处,需要
\begin{equation}
  y(k_I)_y=y(k_T)_y=y(k_R)_y
\end{equation}
可以变换坐标系使入射波在$xOz$平面上,此时$(k_I)_y=0$,相应的其余两项也为0,这意味着入射波,反射波,折射波在同一平面内入射波的法线也在此面内(\textbf{几何光学第一定律}),将这个条件代入(6.36),可以得到
\begin{equation}
  (k_I)_x=(k_T)_x=(k_R)_x
\end{equation}
或
\begin{equation}
  \sin\theta_I=\sin\theta_R=\frac{v_1}{v_2}\sin\theta_T=\frac{n_2}{n_1}\sin\theta_T
\end{equation}
这就导出了\textbf{第二定律}
\begin{equation}
 \theta_I=\theta_R
\end{equation}
即入射角等于反射角和\textbf{第三定律}
\begin{equation}
  \frac{n_1}{n_2}=\frac{\sin\theta_T}{\sin\theta_I}
\end{equation}

除此之外,利用边界条件,可以得到$E_{0_T},E_{0_R}$和$E_{0_I}$之间的关系

\begin{equation}
{E}_{0_R}=\left(\frac{\alpha-\beta}{\alpha+\beta}\right) {E}_{0_I}, \quad {E}_{0_T}=\left(\frac{2}{\alpha+\beta}\right) {E}_{0_I}
\end{equation}
其中
\begin{equation}
\alpha \equiv \frac{\cos \theta_T}{\cos \theta_I}=\frac{\sqrt{1-\left[\left(n_1 / n_2\right) \sin \theta_I\right]^2}}{\cos \theta_I}\quad \beta \equiv \frac{\mu_1 v_1}{\mu_2 v_2}=\frac{\mu_1 n_2}{\mu_2 n_1}
\end{equation}
当垂直入射时,$\theta_I=0$,此时$\alpha=1$,
\begin{equation}
{E}_{0_R}=\left(\frac{1-\beta}{1+\beta}\right) {E}_{0_I}, \quad {E}_{0_T}=\left(\frac{2}{1+\beta}\right) {E}_{0_I}
\end{equation}
除此之外,我们可以预期$\alpha=\beta$时,${E_0}_R=0$,没有反射,全部折射了,此时
\begin{equation}
  \frac{\sqrt{1-\left[\left(n_1 / n_2\right) \sin \theta_I\right]^2}}{\cos \theta_I}=\frac{n_2}{n_1}
\end{equation}
(因为$\mu_1,\mu_2$一般相差不大,所以这里忽略了)
\begin{equation}
  1-\frac{n_1^2}{n_2^2}\sin^2\theta_I=\frac{n_2^2}{n_1^2}\cos^2\theta_I=\frac{n_2^2}{n_1^2}-\frac{n_2^2}{n_1^2}\sin^2\theta_I
\end{equation}
\begin{equation}
  \sin^2\theta_I=\frac{\beta^4-\beta^2}{\beta^4-1}=\frac{\beta^2}{\beta^2+1}
\end{equation}
\begin{equation}
  \cos^2\theta_I=1-\sin^2\theta_I=\frac{1}{\beta^2+1}
\end{equation}
于是,在$\theta_I$满足
\begin{equation}
  \tan\theta_I=\beta=\frac{n_2}{n_1}
\end{equation}
时,会发生无反射的现象,此时$\theta_I=\theta_B$被称为布儒斯特角。

\section{Electromagnetic Waves in Conductors}
现在我们来考虑电磁波入射到导体中的情形。在导体中,我们不能像处理之前在介质中的情形将源都抹掉,导体中是存在自由电流$\J_f$的,为了写出关于$\E,\B$自身的方程,我们利用欧姆定律:$\J=\sigma \vb*{E}$将Maxwell方程组改写为
\begin{align}
        \begin{cases}
                \div{\vb*{E}}=0\text{(理想导体内部的自由电荷会很快的移动到表面)}\\
                \curl{\vb*{E}}=-\pdv{\vb*{B}}{t} \\
                \div{\vb*{B}}=0\\
                \curl{\vb*{B}}=\mu_0\sigma\vb*{E}+\mu_0\epsilon_0\pdv{\vb*{E}}{t}& 
        \end{cases}
\end{align}
这里不像之前那样简单了,退偶后会有
\begin{equation}
\nabla^2 \vb*{E}=\mu \epsilon \frac{\partial^2 \vb*{E}}{\partial t^2}+\mu \sigma \frac{\partial \vb*{E}}{\partial t}
\end{equation}
的形式,但你会发现单色平面波仍然是他的解,只不过此时需要
\begin{equation}
  (i\tilde{k})(i\tilde{k})=\mu\epsilon(-i\omega)(-i\omega)+\mu\sigma(-i\omega)
\end{equation}
即
\begin{equation}
  -\tilde{k}^2=-\mu\epsilon\omega^2-i\mu\sigma\omega
\end{equation}
即在这里的波矢应当是复的,可以写作
\begin{equation}
  \tilde k = k + i\kappa
\end{equation}
可以解得
\begin{equation}
k \equiv \omega \sqrt{\frac{\epsilon \mu}{2}}\left[\sqrt{1+\left(\frac{\sigma}{\epsilon \omega}\right)^2}+1\right]^{1 / 2}, \quad \kappa \equiv \omega \sqrt{\frac{\epsilon \mu}{2}}\left[\sqrt{1+\left(\frac{\sigma}{\epsilon \omega}\right)^2}-1\right]^{1 / 2} .
\end{equation}
实际上由于$i^2=-1$,$\kappa$并不是一个相因子了,而是体现振幅,其暗含了一个指数衰减项:
\begin{equation}
  \vb*{E}(z,t)=\E_0e^{-\kappa z}e^{i(kz-\omega t)}
\end{equation}
磁场也一样,这表明电场在导体内迅速衰减到趋近于0,对于理想导体,$\sigma\rightarrow\infty$,所以$k,\kappa\rightarrow\infty$,电场会在非常短的距离下衰减到0,可以定义趋肤深度
\begin{equation}
  d=\frac{1}{\kappa}
\end{equation}
这体现了电磁波可以进入导体的深度。通过将解代入Maxwell方程组,可以得到和以前一样的结果:电磁波是横波、电场和磁场相互垂直、磁场的振幅是电场的$\frac{\tilde k}{\omega}$,如果规定电场振动方向是x轴,则
\begin{align}
\begin{cases}
		\widetilde{\vb*{E}}(z, t)={E}_0 e^{-\kappa z} e^{i(k z-\omega t)} \hat{\vb*{x}} \\
\widetilde{\vb*{B}}(z, t)=\frac{\tilde{k}}{\omega} E_0 e^{-\kappa z} e^{i(k z-\omega t)} \hat{\vb*{y}}
\end{cases}
\end{align}
可以利用指数形式来表示复数$\tilde{k}$,令
\begin{equation}
  \tilde{k}=Ke^{i\phi}
\end{equation}
其中
\begin{align}
\begin{cases}
K \equiv|\tilde{k}|=\sqrt{k^2+\kappa^2}=\omega \sqrt{\epsilon \mu \sqrt{1+\left(\frac{\sigma}{\epsilon \omega}\right)^2}} \\
\phi \equiv \tan ^{-1}(\kappa / k) .
\end{cases}
\end{align}
所以
\begin{equation}
  \widetilde{\vb*{B}}(z, t)=\frac{K}{\omega} E_0 e^{-\kappa z} e^{i(k z-\omega t+\phi)} \hat{\vb*{y}}
\end{equation}
从指数因子可以看出,电场和磁场现在并不是保持同相了,磁场会比电场附加一个相位$\phi$,假设电场有初相位$\delta_E$,则可以把解写为
\begin{align}
\begin{cases}
\vb*{E}(z, t)=\E_0 e^{-\kappa z} \cos \left(k z-\omega t+\delta_E\right) \hat{\vb*{x}} \\
\vb*{B}(z, t)=\B_0 e^{-\kappa z} \cos \left(k z-\omega t+\delta_E+\phi\right) \hat{\vb*{y}}
\end{cases}
\end{align}
\newpage
\section{Guided Waves}
\subsection{Wave Guides}
在研究了自由空间的电磁波后,我们可以来研究被一圈导体束缚时电磁波的传输,如图所示是一个中空波导
\begin{figure}[H]
\centering
  \includegraphics[width=5cm]{Wave Guide.png}
  \caption{中空波导示例(边界是导体)}
\end{figure}
由于理想导体内没有电场和磁场(趋肤深度无穷小),所以在边界上有边界条件(假设波导内部为真空):
\begin{align}
	\begin{cases}
		\epsilon_0\vb*{E_1}^\perp-\epsilon_2\vb*{E_2}^\perp=\sigma_f\\
		\vb*{E_1}^\parallel=\vb*{E_2}^\parallel\\
		\vb*{B_1}^\perp=\vb*{B_2}^\perp\\
		\frac{\vb*{B_1}^\parallel}{\mu_0}-\frac{\vb*{B_2}^\parallel}{\mu_2}=\vb*{K}_f\times \vb*{\hat n}
	\end{cases}
	\xLongrightarrow[]{\E_2=\B_2=0}
		\begin{cases}
		\epsilon_0\vb*{E_1}^\perp=\sigma_f\\
		\vb*{E_1}^\parallel=0\\
		\vb*{B_1}^\perp=0\\
		\frac{\vb*{B_1}^\parallel}{\mu_0}=\vb*{K}_f\times \vb*{\hat n}
	\end{cases}
\end{align}
而在波导内部的Maxwell方程组为
\begin{align}
        \begin{cases}
                \div{\vb*{E}}=0\\
                \curl{\vb*{E}}=-\pdv{\vb*{B}}{t} \\
                \div{\vb*{B}}=0\\
                \curl{\vb*{B}}=\frac{1}{c^2}\pdv{\vb*{E}}{t}& 
        \end{cases}
\end{align}
这个方程的解仍然可以写作平面波
\begin{align}
	    \vb*{E}=\vb*{E}_0e^{\i(kz-\omega t)}\\
  \B=\B_0e^{\i(kz-\omega t)}	\notag
\end{align}
但是我们需要注意的是,事实上电磁波在波导内的传输并不是沿着z轴的,其会在垂直于z轴方向产生驻波,最终看起来的能量传输方向是沿z轴的,所以如果定义$\E,\B$是沿z轴的平面波,其\textbf{形式}不一定是横波(这里可能说的不是很明白,但是我尽力了!!可以去查一下资料,实际上后面说的TE、TM波仍然是横波!所谓的TE、TM波的“纵波”成分也是驻波和行波的叠加结果。不要产生误解以为波导内的电磁波就是纵波了!)所以我们令
\begin{equation}
\begin{aligned}
	  \E_0=\E_0(x,y)=E_x(x,y)\x+E_y(x,y)\y+E_z(x,y)\z\\
	  \B_0=\B_0(x,y)=B_x(x,y)\x+B_y(x,y)\y+B_z(x,y)\z
\end{aligned}
\end{equation}
将其代回(6.55)并代入Maxwell方程组的2,4会得到以下六个方程
\begin{align}
	\begin{cases}
		\frac{\partial E_y}{\partial x}-\frac{\partial E_x}{\partial y}=i \omega B_z\\
		\frac{\partial E_z}{\partial y}-i k E_y=i \omega B_x \\
		i k E_x-\frac{\partial E_z}{\partial x}=i \omega B_y\\
		\frac{\partial B_y}{\partial x}-\frac{\partial B_x}{\partial y}=-\frac{i \omega}{c^2} E_z\\
		\frac{\partial B_z}{\partial y}-i k B_y=-\frac{i \omega}{c^2} E_x\\
		i k B_x-\frac{\partial B_z}{\partial x}=-\frac{i \omega}{c^2} E_y\\
	\end{cases}
\end{align}
我们下一步要做的是用$E_z,B_z$来表示其余的几个分量来研究$E_z,B_z$的分布

\begin{align}
\begin{cases}
E_x & =\frac{i}{(w / c)^2-k^2}\left(k \frac{\partial E_z}{\partial x}+w \frac{\partial B_z}{\partial y}\right)\\
E_y&=\frac{i}{(w / c)^2-k^2}\left(k \frac{\partial E_z}{\partial y}-w \frac{\partial B_z}{\partial x}\right) \\
B_x & =\frac{i}{(w / c)^2-k^2}\left(k \frac{\partial B_z}{\partial x}-\frac{w}{c^2} \frac{\partial E_z}{\partial y}\right)\\
B_y&=\frac{i}{(w / c)^2-k^2}\left(k \frac{\partial B_z}{\partial y}+\frac{w}{c^2} \frac{\partial E_z}{\partial x}\right) 
\end{cases}
\end{align}
再将其带入Maxwell方程组的1、3式,可以得到关于$E_z$,$B_z$的方程
\begin{equation}
\begin{aligned}
& {\left[\frac{\partial^2}{\partial x^2}+\frac{\partial^2}{\partial y^2}+(w / c)^2-k^2\right] E_z=0} \\
& {\left[\frac{\partial^2}{\partial x^2}+\frac{\partial^2}{\partial y^2}+(w / c)^2-k^2\right] B_z=0}
\end{aligned}
\end{equation}
当$E_z=0$时,说明传输方向E以横波形式传播,称为TE波,反之,当$B_z=0$时,说明传输方向B以横波形式传播,称为TM波,如果$E_z=B_z=0$,则为TEM波。
\begin{figure}[H]
\centering
  \includegraphics[width=10cm]{TETMwave.png}
  \caption{TE,TM波图示}
\end{figure}
在中空波导中不能传输TEM波。因为若存在TEM波,则B线是一个垂直于传输方向的回路(可以想象为6.6中的TM波的红色框,但此时电场线为TE波的蓝色线),由安培定理的积分形式:
\begin{equation}
  \oint \B\cdot \dd \vb*{l}=\mu_0I_d
\end{equation}
其中$I_d$为传输方向的位移电流,但很显然电场是横向的,不会有纵向(传输方向)的位移电流,所以$\vb*{B}$是无旋场,这是矛盾的,这表明$B$并不存在,于是中空波导不能传输TEM波。
\subsection{TE Waves in a Rectangular Wave Guide}
下面我们来研究中空波导中的TE波。
\begin{figure}[H]
\centering
  \includegraphics[width=5cm]{TE Wave.png}
  \caption{波导示意图}
\end{figure}
从(6.63)的第三式可以导出边界条件为
\begin{align}
	\begin{cases}
		B_x(0,y)=B_x(a,y)=0\\
		B_y(x,0)=B_y(x,b)=0
	\end{cases}
\end{align}
可以将$B_z(x,y)$分离变量
\begin{equation}
  B_z(x,y)=X(x)Y(y)
\end{equation}
使得
\begin{align}
\begin{cases}
	\frac{1}{X} \frac{\dd^2 X}{\dd x^2}=-k_x^2\\
	\frac{1}{Y} \frac{\dd^2 Y}{\dd y^2}=-k_y^2
\end{cases}
\end{align}
其满足方程(6.69),即
\begin{equation}
-k_x^2-k_y^2+(\omega / c)^2-k^2=0
\end{equation}
(6.73)有分离变量解:
\begin{align}
\begin{cases}
X(x)=A \sin \left(k_x x\right)+B \cos \left(k_x x\right)\\
Y(y)=C \sin \left(k_y y\right)+D \cos \left(k_y y\right)
\end{cases}
\end{align}
为了定出系数和k的可能值,需要代入边界条件(6.71),而(6.71)需要带入$B_{x,y}$关于$E_{z},B_{z}$的方程(注意这里$E_{z}=0$),作为示例,我将带入一个:考虑$B_x(0,y)=0$,则
\begin{equation}
  0=\pdv{B_z(0,y)}{x}=X'(0)Y(y)\Rightarrow X'(0)=0
\end{equation}
以此类推,可以得到
\begin{equation}
k_x=\frac{m \pi}{a},k_y=\frac{n \pi }{b}, A=C=0
\end{equation}
所以$k_x,k_y$并不能随意取值,这里的$k_x,k_y$被称为其本征值,代入原函数可以得到
\begin{equation}
B_z=B_0 \cos (\frac{m \pi x}{a}) \cos (\frac{n \pi y}{b})
\end{equation}
这被称为$TE_{mn}$模式。
可以把$k_{x,y}$代入(6.74)来得到$k$:
\begin{equation}
\begin{aligned}
& k=\sqrt{\left(\frac{\omega}{c}\right)^2-\pi^2\left[\left(\frac{m}{a}\right)^2+\left(\frac{n}{b}\right)^2\right]} \\
& =\frac{1}{c} \sqrt{w^2-w_{m n}^2}
\end{aligned}
\end{equation}
当然,对于一个给定的mn,如果$\omega<\omega_{mn}$则k会变成复数,可以写成$\tilde k=k+i\kappa$,此时,$i\kappa$会和相因子的$i$相乘,变成e指数衰减而导致场在波导内迅速衰减到0(和在金属中一样),所以$\omega_{mn}$称为mn模式的截止频率。对于最小的$TE_{10}$,可以给定最低阶段频率
\begin{equation}
  \omega_{10}=\frac{c\pi}{a}
\end{equation}
这由波导的几何结构决定。

\chapter{Potentials and Fields}
我们在静电磁学中曾讨论过关于势和场的话题,而现在,我们希望在电动力学中找到Maxwell方程组对应的势和场。
\section{Potentials}
从Maxwell方程组出发(这里使用真空中的):
\begin{align}
        \begin{cases} 
                \div{\vb*{E}}=\frac{\rho}{\epsilon_0}\\ 
                \curl{\vb*{E}}=-\pdv{\vb*{B}}{t} \\
                \div{\vb*{B}}=0\\
                \curl{\vb*{B}}=\mu_0\vb*{J}+\mu_0\epsilon_0\pdv{\vb*{E}}{t}& 
        \end{cases}
\end{align}
注意到这里的$\E$的旋度不再为0,于是我们不能像在静电场中那样建立标量势,但仍然有$\div\B=0$,于是可以定义
\begin{equation}
  \curl{\A}=\B
\end{equation}
将其带入(7.1)的式2,则
\begin{equation}
  \curl{\left(\E+\pdv{\A}{t}\right)}=0
\end{equation}
所以$\E+\pdv{\A}{t}$是无旋的,可以写作一个标量场的梯度,这样我们可以重新定义在电动力学中的标量势
\begin{equation}
  \E+\pdv{\A}{t}=-\nabla V
\end{equation}
(7.1)的式1、4还没有利用过,可以将(7.2)、(7.4)代入,有
\begin{align}
	\nabla^2V+\pdv{}{t}\div{\A}=-\frac{\rho}{\epsilon_0}
\end{align}
以及
\begin{equation}
  \begin{aligned}
	  \curl \curl \A=\nabla(\div{\A})-\nabla^2\A&=\mu_0\vb*{J}+\mu_0\epsilon_0\pdv{\vb*{E}}{t}	\\
	  &=\mu_0\vb*{J}-\mu_0\epsilon_0\left(\pdv[2]{\A}{t}+\grad\left(\pdv{V}{t}\right)\right)
\end{aligned}
\end{equation}
整理有
\begin{equation}
  \mu_0\epsilon_0\pdv[2]{\A}{t}-\nabla^2\A+\nabla\left(\div\A+\mu_0\epsilon_0\pdv{V}{t} \right)=\mu_0\J
\end{equation}


\subsection{Gauge Transformations}
因为场是可观测量,而势不是,势是人为规定的一个量,所以有一定的自由度来定义有着不同性质的势,但需要确保他们对应着的场是一样的,这就是规范变化。我们可以看到上面的式子略显复杂,所以一个自然的想法就是我们能不能通过某种方法来将这些关于势的方程进行简化,这需要用到规范变换。

首先从矢量势出发,$\curl\A=\B$,所以在$A$的基础上增加一个无旋量,得到的$\B$不会改变,注意到梯度场是无旋的,所以可以定义
\begin{equation}
  \A'=\A+\nabla\lambda
\end{equation}
代入(7.5),由于$\rho$是相同的,所以需要
\begin{equation}
\begin{aligned}
	     \nabla^2 V+\pdv{}{t}\div \A&=\nabla^2 V'+\pdv{}{t}\div \A'\\
	     &=\nabla^2 V'+\pdv{}{t}\div \A+\pdv{}{t}\nabla^2\lambda	\\
\end{aligned}
\end{equation}
这意味着
\begin{equation}
  \laplace V'=\laplace\left(V-\pdv{\lambda}{t}\right)
\end{equation}
或
\begin{align}
\begin{cases}
	    \A'=\A+\nabla\lambda\\
	    V'=V-\pdv{\lambda}{t}
\end{cases}
\end{align}
所以可以取这样的$\lambda$作如(7.11)这样的变换,此时势对应的场是一致的。

\subsection{Coulomb Gauge and Lorenz Gauge}
有了以上的理论基础,我们现在可以着手来简化势函数了。
\begin{align}
\begin{cases}
	  \nabla^2V+\pdv{}{t}\div{\A}=-\frac{\rho}{\epsilon_0}\\
	  \mu_0\epsilon_0\pdv[2]{\A}{t}-\nabla^2\A+\nabla\left(\div\A+\mu_0\epsilon_0\pdv{V}{t} \right)=\mu_0\J
\end{cases}
\end{align}

在静电磁学中,我们曾介绍过Coulomb规范(3.20),即$\div\A=0$,这会使静电磁学中的$V,\A$都变成Poisson方程的形式,在那里也给出了$V,\A$的解,不过在电动力学中,如果仍然采用Coulumb规范,关于V的方程会回到Poisson方程,但关于$\A$的方程仍然是不好解的。

所以在电动力学中,一个更方便的形式是Lorenz规范:
\begin{equation}
  \div\A+\mu_0\epsilon_0\pdv{V}{t}=0
\end{equation}
此时(7.12)不但完成了简化,还完成了退偶!现在我们得到的是关于$V,\A$自身的方程了:
\begin{align}
\begin{cases}
	    \laplace V-\mu_0\epsilon_0\pdv[2]{V}{t}=-\frac{\rho}{\epsilon_0}\\
	    \laplace \A-\mu_0\epsilon_0\pdv[2]{\A}{t}=-\mu_0\J
\end{cases}
\end{align}
他们在形式上也是相同的!可以定义达朗贝尔算符:
\begin{equation}
  \square^2=\laplace -\mu_0\epsilon_0\pdv[2]{}{t}
\end{equation}
这样(7.14)可以写作一个简单的形式:
\begin{align}
\begin{cases}
	  \square^2V=-\frac{\rho}{\epsilon_0}\\
	  \square^2\A=-\mu_0\J
\end{cases}
\end{align}
\subsection{Solution - Retarded Potentials}
下面我们就是要给出(7.14)的解。实际上$(7.14)$是三维无界空间的受迫振动问题,
整理为标准形式:
\begin{equation}
 c^2\laplace V+\frac{c^2\rho}{\epsilon_0}=\pdv[2]{V}{t}
\end{equation}
可以使用积分变换法来求解,使用Laplace+Fourier变换(时间维度进行Laplace变换,空间维度进行Fourier变换),可得到
\begin{equation}
  V(\vb*{r},t)=\frac{1}{4\pi c^2}\iiint\left[\int_0^t\frac{c^2\rho(\vb*{r'},\tau)}{\epsilon_0 \rr}\delta(\tau-t+\frac{\rr}{c})\dd \tau\right]\dd \vb*{r}'
\end{equation}
由于$\delta$函数的选择性,只有
\begin{equation}
  \tau=t-\frac{\rr}{c}
\end{equation}
时有解,即
\begin{equation}
  V(\vb*{r},t)=\frac{1}{4\pi c^2}\iiint\frac{c^2\rho(\vb*{r'},t-\frac{\rr}{c})}{\epsilon_0 \rr}\dd \vb*{r}'
\end{equation}
或
\begin{equation}
    V(\vb*{r},t)=\frac{1}{4\pi\epsilon_0}\iiint\frac{\rho(\vb*{r'},t-\frac{\rr}{c})}{ \rr}\dd \vb*{r}'
\end{equation}
所以事实上$V$在$t$时刻的值由推迟时间$t_r=t-\frac{\rr}{c}$所对应的空间电荷分布决定,这被称为推迟势。实际上,$\vb*{r}'$虽然是对全空间积分,但是我们可以把积分面缩小:注意到式(7.19),并注意到$0<\tau<t$,这次我们从空间的角度来看,则有
\begin{equation}
\begin{aligned}
	  0&<t-\frac{\rr}{c}<t\\
	  0&<\frac{\rr}{c}<t\\
	  0&<\rr<ct
\end{aligned}
\end{equation}
这个式子的意思是只有在$0<\rr<ct$时积分才不为0,所以我们可以限定积分区域为
\begin{equation}
  \abs{\vb*{r-r'}}<ct
\end{equation}
这是一个球,在t时刻只有这个球内的电荷分布是有贡献的!实际上这很好理解,因为在真空中电磁场以光速传播,所以只有在以$ct$为半径的球的范围内的扰动才对这一点有贡献。

(这个部分的内容可以参考「数学物理方法」,在上面有更详细的阐述。)

矢量势的解也是类似的:
\begin{equation}
  \A(\vb*{r},t)=\frac{\mu_0}{4\pi}\iiint\frac{\J(\vb*{r'},t-\frac{\rr}{c})}{ \rr}\dd \vb*{r}'
\end{equation}

\section{A Moving Point Charge}
下面我们来研究最简单的情况:一个移动的点电荷的势和场。假设点电荷以$\vb*{w}(t)$轨迹运动,如下图
\begin{figure}[H]
\centering
  \includegraphics[width=5cm]{Moving Charge.png}
\end{figure}
可以证明在某一时刻,$\vb*{r}$处只能感受到点电荷处于一个位置所产生的势于是此时$t_r$可以表示为
\begin{equation}
  t_r=t-\frac{\abs{\vb*{r}-\vb*{w}(t_r)}}{c}
\end{equation}
所以
\begin{equation}
\begin{aligned}
	    V(\vb*{r},t)=&\frac{1}{4\pi\epsilon_0}\iiint\frac{\rho(\vb*{r'},t-\frac{\rr}{c})}{ \rr}\dd \vb*{r}'	\\
	    \xlongequal[\text{是有效的}]{\text{只有一个点}}&\frac{1}{4\pi\epsilon_0}\frac{1}{\rr}\iiint \rho(\vb*{r'},t-\frac{\rr}{c})\dd \vb*{r'}\\
	    =&\frac{1}{4\pi\epsilon_0}\frac{1}{\rr}\frac{q}{1-\hr\cdot\frac{\vb*{v}}{c}}\\
	    =&\frac{1}{4\pi\epsilon_0}\frac{q}{\rr-\br\cdot\frac{\vb*{v}}{c}}
\end{aligned}
\end{equation}
\begin{equation}
  \A(\vb*{r},t)=\frac{\mu_0}{4\pi}\frac{q\vb*{v}}{\rr-\br\cdot \frac{\vb*{v}}{c}}=\frac{\vb*{v}}{c^2}V(\vb*{r},t)
\end{equation}
注意这里出现的
\begin{equation}
  \frac{1}{1-\hr\cdot\frac{\vb*{v}}{c}}
\end{equation}
因子是由于几何效应导致的,可以用几何来证明运动的物体看起来要长一些,当然你也可以直接计算这个积分得到。

在得到势函数后,可以通过
\begin{align}
\begin{cases}
	   \E=-\nabla V-\pdv{\A}{t}\\
	   \B=\curl{\A}
\end{cases}
\end{align}
来得到电磁场。算的过程太复杂,我这里就不写了哈哈,直接给出结论并说明!电场为
\begin{equation}
\vb*{E}=-\nabla V-\frac{\partial \vb*{A}}{\partial t}=\frac{q}{4 \pi \varepsilon_0} \frac{\rr}{(\br \cdot \vb*{u})^3}\left[\left(c^2-v^2\right) \vb*{u}+\br \times(\vb*{u} \times \vb*{a})\right]
\end{equation}
磁场为
\begin{equation}
\begin{aligned}
\vb*{B} =\nabla \times \vb*{A}&=\frac{1}{c} \frac{q}{4 \pi \varepsilon_0} \frac{1}{(\vb*{u} \cdot \br)^3} \br \times\left[\left(c^2-v^2\right) \vb*{v}+\br \times(\vb*{u} \times \vb*{a})\right] \\
& =\frac{1}{c} \hat{\br} \times \vb*{E}
\end{aligned}
\end{equation}
其中$\vb*{u}=c\hr-\dv{\vb*{w}}{t}=c\hr-\vb*{v}$,可以注意到这个结果中有关于$\vb*{v}=\dv{\vb*{w}}{t}$的项,也有关于$\vb*{a}=\dv[2]{\vb*{w}}{t}$的项,所以一个移动点电荷所产生的场有关于速度的场,也有关于加速度的场,对于静电情况($\vb*{v}=\vb*{a}=0$,$\vb*{u}=c\hr$),有
\begin{equation}
  \vb*{E}=\frac{q}{4\pi\epsilon_0}\frac{\rr}{c^3\rr^3}c^3\hr=\frac{q}{4\pi\epsilon_0}\frac{1}{\rr^2}\hr
\end{equation}
\begin{equation}
  \B=0
\end{equation}
这正是我们在这门课刚开始时使用库伦定律规定的点电荷所产生的电场!而由于静止点电荷不会产生电流,所以磁场强度为0。

\chapter{Radiation}
在这一章中,我们来研究电磁波的产生,也就是如何产生电磁辐射。

产生电磁辐射就要求在自由空间的无限远处,仍然有电磁场能量的传输,可以用方程写作
\begin{equation}
  \lim_{r\rightarrow\infty}\oiint \vb*{S}\cdot \dd \vb*{a}=\lim_{r\rightarrow\infty}\frac{1}{\mu_0}\oiint \E\times\B\cdot \dd \vb*{a}=\text{finite}
\end{equation}
这要求$\E\times \B$在远处的衰减不能快于$\frac{1}{r^2}$,考虑到推迟势所对应的场
\begin{equation}
\begin{gathered}
\vb*{E}\left(\vb*{r}_0, t\right)=\frac{1}{4 \pi \varepsilon_0} \int\left(\frac{\rho\left(\vb*{r}^{\prime}, t_r\right)}{\rr^2} \hr+\frac{\dot{\rho}\left(\vb*{r}^{\prime}, t_r\right)}{c \rr} \hr-\frac{\dot{\vb*{J}}\left(\vb*{r}^{\prime}, t_r\right)}{c^2 \rr}\right) \dd \tau^{\prime} \\
\vb*{B}\left(\vb*{r}_0, t\right)=\frac{\mu_0}{4 \pi} \int\left(\frac{\vb*{J}\left(\vb*{r}^{\prime}, t_r\right)}{\rr^2}+\frac{\dot{\vb*{J}}\left(\vb*{r}^{\prime}, t_r\right)}{c \rr}\right) \times \hr \dd \tau^{\prime}
\end{gathered}
\end{equation}
所以实际上只有$\dot \rho,\dot\J$对辐射有贡献,所以静电荷、稳恒电流是无法产生辐射的。

下面来看两种简单的情形,一种是电偶极子辐射,另一种是磁偶极子辐射,在本章的最后,将考虑点电荷产生的辐射。因为这部分比较复杂,我就不敲到\LaTeX 里了,我把我的手写版拍成图片附在此页后。
\section{Dipole Radiation}
\begin{figure}[H]
\centering
  \includegraphics[width=15cm]{Radiation1.jpg}
\end{figure}
\begin{figure}[H]
\centering
  \includegraphics[width=15cm]{Radiation2.jpg}
\end{figure}

\section{Point Charge Radiation}

移动点电荷所产生的场由$(7.30),(7.31)$给出,考虑辐射条件:$\lim_{r\rightarrow\infty}\frac{1}{\mu_0}\oiint \vb*{S}\cdot \dd \vb*{a}=\text{finite}$,首先来计算玻印廷矢量:
\begin{equation}
  \vb*{S}=\frac{1}{\mu_0}\E\times\B=\frac{1}{\mu_0c}\E\times\hr\times\E=\frac{1}{\mu_0c}\left[\hr(\E\cdot\E)-\E(\E\cdot\hr)\right]
\end{equation}
其中,
\begin{equation}
  \vb*{E}=\frac{q}{4 \pi \varepsilon_0} \frac{\rr}{(\br \cdot \vb*{u})^3}\left[\left(c^2-v^2\right) \vb*{u}+\br \times(\vb*{u} \times \vb*{a})\right]
\end{equation}
而这里的第一项$\propto 1/r^2$在平方后以四次方形式衰减,所以对辐射没有贡献,对辐射有贡献的是
\begin{equation}
  \E_{rad}=\frac{q}{4 \pi \varepsilon_0} \frac{\rr}{(\br \cdot \vb*{u})^3}\br \times(\vb*{u} \times \vb*{a})\perp\hr
\end{equation}
\begin{equation}
  \vb*{S}=\frac{1}{\mu_0c}\left[E_{rad}^2\hr-(\E_{rad}\cdot\hr)\E_{rad}\right]=\frac{1}{\mu_0c}E_{rad}^2\hr
\end{equation}
一般情况下,还是比较复杂,下面考虑一个特殊情况:$v=0$,即在某一时刻速度为0但加速度不为0的情况,此时$\vb*{u}=c\hat{\rr}-\vb*{v}=c\hr$
\begin{equation}
\begin{aligned}
  \E_{rad}&=\frac{q}{4\pi\epsilon_0}\frac{1}{c^2\rr}\hr\times(\hr\times \vb*{a})\\
  &=\frac{1}{4\pi\epsilon_0}\frac{q}{c^2\rr}(\hr(\hr\cdot \vb*{a})-\vb*{a}(\hr\cdot\hr))\\
  &=\frac{1}{4\pi\epsilon_0}\frac{q}{c^2\rr}(\hr(\hr\cdot \vb*{a})-\vb*{a})
\end{aligned}
\end{equation}
而
\begin{equation}
\begin{aligned}
	    \vb*{S}=\frac{1}{\mu_0c}\E_{rad}^2\hr&=\frac{1}{\mu_0c}\left(\frac{1}{4\pi\epsilon_0}\frac{q}{c^2\rr}\right)^2((\hr\cdot \vb*{a})^2+\vb*{a}^2-2\vb*{a}\cdot\hr(\hr\cdot \vb*{a}))\hr	\\
	    &=\frac{\mu_0q^2}{16\pi^2c\rr^2}(a^2-(\hr\cdot \vb*{a})^2)\\
	    &=\frac{\mu_0q^2a^2}{16\pi^2c\rr^2}\sin^2\theta \hr
\end{aligned}
\end{equation}
其中$\theta$是$\vb*{a}$和$\hr$之间的夹角。所以辐射的图像是运动的点电荷会在垂直于运动方向产生一个环型辐射,在沿着运动方向并没有辐射,如下图\begin{figure}[H]
\centering
  \includegraphics[width=5cm]{Point Radiation.png}
\end{figure}最后,我们还可以求得总辐射功率:
\begin{equation}
  p=\oiint \vb*{S}\cdot\dd \vb*{a}=\frac{\mu_0 q^2 a^2}{6\pi c}
\end{equation}


\chapter{Electrodynamics and Relativity}

\section{Special Relativity}
\subsection{Einstein's Postulates}
\begin{figure}[H]
\centering
  \includegraphics[width=5cm]{relativity.png}
\end{figure}
如上图,两人正在一辆移动的列车上拿着线圈通过一个有磁场的区域,在我们地面参考系看来,这个过程实际上是在穿过磁场时,线圈切割磁场线,产生了动生电动势,这个电动势来源于运动的电子收到磁场力作用形成的。但如果从另一个角度来看,事情好像发生了一些变化:如果我们时车上的人,或者说从线圈自身角度来看,它只感觉到它所在的区域的磁场强度发生了变化,相应的磁通量发生变化,由此产生了感生电动势,但这一过程是由于$\curl\E=-\pdv{\B}{t}$,其机理是产生了一个感生电场,电动势的产生在这里又是一个电的过程了。但是无论如何,他们都可以从法拉第电磁感应定律导出相同的电动势:
\begin{equation}
  \epsilon=-\dv{\Phi}{t}
\end{equation}
其中,$\Phi$是磁通量。在经典理论里,这两种解释不可能同时成立,但爱因斯坦认为,实际上这两种说法是无关紧要的,都是成立的,据此和一些为了解释上述现象的“以太”论的失败,爱因斯坦提出了两条假设:
\begin{enumerate}
  \item \textbf{相对性原理}:物理定律在所有惯性系中都是适用的。
  \item \textbf{光速不变原理}:真空中的光速对于所有惯性系中的观察者都是相同的,无论光源有没有移动。
\end{enumerate}
事实上,这两条假设直接导出了狭义相对论。

\subsection{The Geometry of Relativity}
下面我们通过爱因斯坦的两条假设来推出相对论在几何上的一些性质:
\subsubsection{The Relativity of Simultaneity}
相对论会导致“同时”的相对性,也就是某两个事件在某一惯性参考系(下面所说的“参考系”无特殊说明则均指惯性系)下所观察到是同时的,而在另一个参考系下可能并不同时。
\begin{figure}[H]
\centering
  \includegraphics[width=10cm]{georel1.png}
\end{figure}
如图所示,一盏灯悬挂在平稳运行的列车的正中央,在列车内的角度上来看,事件(a):光到达前方和事件(b):光到达后方是同时发生的。但在地面上来看,事情似乎不是这样的,在地面参考系中,由于列车的运动,导致在某时刻发出的光,需要抵达后方的距离减小,时间缩短。(注意这里应用了光速不变这一假设,这导致在地面参考系下光速仍然是$c$而不是$c\pm v$,否则你无法得到这一结论)

\subsubsection{Time Dilation}
相对论会导致运动的时钟变慢,这就是时间膨胀效应。为了证明他,我们来考虑这样一个过程:
\begin{figure}[H]
\centering
  \includegraphics[width=6cm]{georel2.png}
\end{figure}
考虑在一个运行的列车上,一盏灯垂直射向火车的地板上,在车厢内部的人看来,光运行的路径是竖直向下的一条线,长度为车厢高度h,于是在车厢内的人会给出:(注意从此以后在某个量上带一个bar的均指在运动的参考系中测量的结果)
\begin{equation}
  h=c\Delta\bar{t}
\end{equation}
但在地面参考系上看,光在某时刻出射以后经过一段时间射向了地板,这个过程中,火车运行了一段位移$v\Delta t$(由于是在地面参考系上来看,所以这里没有bar),于是有
\begin{equation}
  h^2+v^2\Delta t^2=c^2\Delta t^2
\end{equation}
即
\begin{equation}
  \Delta t=\frac{h}{c}\frac{1}{\sqrt{1-v^2/c^2}}=\frac{1}{\sqrt{1-v^2/c^2}}\Delta \bar t
\end{equation}
可以令$\gamma=\frac{1}{\sqrt{1-v^2/c^2}}$(注意因为$v\leq c$所以$\gamma\geq 1(\text{仅当}v=0\text{时取到等号})$)
所以
\begin{equation}
  \Delta \bar t=\frac{1}{\gamma}\Delta t
\end{equation}
这说明一个运动的物体对两个事件发生的时间间隔将变短,比例为$\frac{1}{\gamma}$。

\subsubsection{Lorentz Contraction}
相对论的另一个几何效应就是从静止参考系来看,运动的物体变短了。考虑在一个运动的车厢内从一边点亮一盏灯,由于对面的镜子,光会反射回来,考虑事件:光发出至反射回车厢左面。
\begin{figure}[H]
\centering  %图片全局居中
\subfigure[车内参考系]{
\label{Fig.lor.1}
\includegraphics[width=0.3\textwidth]{georel3}}
\subfigure[地面参考系]{
\label{Fig.lor.2}
\includegraphics[width=0.3\textwidth]{georel4}}
\caption{Lorentz Contraction}
\label{Fig.lor}
\end{figure}
从图\ref{Fig.lor.1}中可以看到,在车内,如果令静止的车厢长度为$l$,则可以列出
\begin{equation}
  2\bar l=c\Delta \bar t
\end{equation}
而在地面参考系中,情况如\ref{Fig.lor.2}所示,由于车厢的运动,光从车厢的左边照向镜子,所行走的路程是$l+v\Delta t_1$,而反射回来走了$l-v\Delta t_2$,其中$\Delta t_1+\Delta t_2=\Delta t$,为在地面参考系下的总时长,则有
\begin{equation}
  \Delta t_1=\frac{l}{c-v}\quad,\quad\Delta t_2=\frac{l}{c+v}
\end{equation}
而$\Delta t=\Delta t_1+\Delta t_2$,有
\begin{equation}
 \gamma\Delta\bar t= \Delta t=2\frac{l}{c}\frac{1}{1-v^2/c^2}=\frac{2l}{c}\gamma^2
\end{equation}
结合式(9.6),有
\begin{equation}
  \frac{2\bar l}{c}\gamma=\frac{2l}{c}\gamma^2
\end{equation}
即
\begin{equation}
  \bar l=\gamma l
\end{equation}
所以与车厢内测量的结果比较,地面上的观察者会认为车厢的长度要小一个因子$\frac{1}{\gamma}$,这就是洛伦兹收缩。

\subsection{Lorentz Transformation}
在有了这两个几何性质以后,我们可以来探索在相对论下一个事件发生时在不同参考系中的一般情况了。首先我们来定义一个事件,一个事件在某一个参考系中有确定的时间和空间,在某个瞬间在某处完成,可以由其在该参考系中发生的位置和时刻来表示,由此可以定义一组时空坐标$(x,y,z,t)$,下面我们就是要来求出在狭义相对论下,这组时空坐标在运动的参考系下该如何表示。为此,我们考察两个参考系,一个是$S$系,而另一个参考系以速度$v$沿$S$系的$x$轴运动($\bar S$),$S$系视角下如图
\begin{figure}[H]
\centering
  \includegraphics[width=7cm]{lorentz1.png}
  \caption{S系视角}
\end{figure}
我们来考虑建立方程,首先在S系下,事件E的时空坐标为$(x,t)$(为了简化这里没有加入y和z)而由于$\bar S$参考系下的运动,E在S参考系的视角下,在$\bar S$参考系的$x$坐标应为
\begin{equation}
  d=x-vt
\end{equation}
但实际上$d$是$S$系视角下的$E$在$\bar S$系的$\bar x$坐标,由于相对论的洛伦兹收缩,这个$d$比实际值要小,真实的$\bar x$坐标应当是
\begin{equation}
  d=\frac{\bar x}{\gamma}=x-vt\Rightarrow\bar x=\gamma(x-vt)
\end{equation}
这导出了$\bar x$与$x,t$的关系,但是我们还想得到关于$t$和$\bar t$之间变换的关系,为此,我们这次在$\bar S$上来看,如下图
\begin{figure}[H]
\centering
  \includegraphics[width=7cm]{Lorentz2.png}
  \caption{$\bar S$系视角}
\end{figure}
这次,我们发现$S$系在以速度$v$向$\bar S$系的$x$轴负方向运动。可以列出
\begin{equation}
  \bar d=\bar x+v\bar t
\end{equation}
$\bar d$是在$\bar S$视角下对$S$的观测,但是$S$系相对$\bar S$系是运动的,所以实际上$\bar d$测小了,真实情况是
\begin{equation}
  \bar d=\frac{x}{\gamma}=\bar x+v\bar t
\end{equation}
注意到代换式9.12有
\begin{equation}
  v\bar t=\frac{x}{\gamma}-\gamma(x-vt)
\end{equation}
即
\begin{equation}
  \bar t=\gamma t+\frac{1-\gamma^2}{\gamma v}x=\gamma(t-\frac{v}{c^2}x)
\end{equation}
于是我们得到了从$S$系变换到以速度$v$沿$S$系的$x$轴运动的$\bar S$系的洛伦兹变换式
\begin{align}
	\begin{cases}
		\bar x=\gamma(x-vt)\\
	  \bar y=y\\
	  \bar z=z\\
	  \bar t =\gamma(t-\frac{v}{c^2}x)
	\end{cases}
\end{align}
可以利用洛伦兹变换来得到爱因斯坦的速度加法律。

我们都知道在非相对论情况下,有$v_{AB}+v_{BC}=v_{AC}$,但在相对论下,这个式子需要进行修正,为此我们考虑一个粒子相对于S系运动速度为$u$,有
\begin{equation}
  u=\dv{x}{t}
\end{equation}
但在以速度$v$沿$S$系的$x$轴运动的$\bar S$系中,根据洛伦兹变换,$\dd x$和$\dd t$都会发生变化,有
\begin{equation}
  \bar u=\frac{\gamma(\dd x-v\dd t)}{\gamma(\dd t-v/c^2\dd x)}=\frac{u-v}{1-uv/c^2}
\end{equation}
注意到如果设粒子为$A$,则有
\begin{equation}
  v_{A\bar S}=\frac{v_{AS}+v_{S\bar S}}{1+v_{AS}v_{S\bar S}/c^2}
\end{equation}
对于一般情况,类似的可以得到
\begin{equation}
  v_{AC}=\frac{v_{AB}+v_{B C}}{1+v_{AB}v_{B C}/c^2}
\end{equation}
这里可以看到当$v_{AB}v_{BC}\ll c^2$时,也就是速度无法和光速比拟时,速度加法律又回到了非相对论情况,这也是符合预期的。

\subsection{4-Vectors and the Structure of Spacetime}
注意到9.17给出的洛伦兹变换式并不是非常的对称,为此我们可以引入4-矢量,即使用坐标:
\begin{equation}
  x^0=ct,\quad x^1=x,\quad x^2=y,\quad x^3=z
\end{equation}
注意这里的上标并不是“次方”,而是表示4-矢量的第某个分量,其中第0分量在$t$的基础上乘上了光速$c$,这样其就有了长度的量纲。在这个表示方法下,并定义$\beta=v/c$,洛伦兹变换式可以改写为
\begin{align}
	\begin{cases}
		\bar x^0=\gamma(x^0-\beta x^1)\\
		\bar x^1=\gamma(x^1-\beta x^0)\\
		\bar x^2=x^2\\
		\bar x^3=x^3
	\end{cases}
\end{align}
当然也可以写作矩阵形式
\begin{equation}
\left(\begin{array}{c}
\bar{x}^0 \\
\bar{x}^1 \\
\bar{x}^2 \\
\bar{x}^3
\end{array}\right)=\left(\begin{array}{cccc}
\gamma & -\gamma \beta & 0 & 0 \\
-\gamma \beta & \gamma & 0 & 0 \\
0 & 0 & 1 & 0 \\
0 & 0 & 0 & 1
\end{array}\right)\left(\begin{array}{l}
x^0 \\
x^1 \\
x^2 \\
x^3
\end{array}\right)
\end{equation}
可以令这个矩阵为洛伦兹变换矩阵,记为$\Lambda$,用$\Lambda_\nu^\mu$来表示$\Lambda$矩阵第$\mu$行第$\nu$列的元素,于是9.24也可以写作
\begin{equation}
  \bar x^\mu=\sum_{\nu}\Lambda^\mu_{\nu}x^\nu\quad(\mu=0,1,2,3)
\end{equation}

事实上,至此已经是狭义相对论的全部知识了(狗头),但是物理学家显然不会满足于找到不同参考系下的变换公式,换句话说,不同参考系下的变换公式好像是在从不同角度来看四维时空中的“事件”,物理学家希望找到这个“事件”本身的某个性质,为此,我们需要找到“事件”变换中的某个不变量。不妨回到狭义相对论的开始,爱因斯坦假设“光速”不变,从这个理论做突破口,现在我们来考虑一个简单的事情,在$t=0$时刻,$S,\bar S$参考系的零点在同一处,一束光从零点出发,在此后的某时刻,对于$S$参考系有
\begin{equation}
  x^2+y^2+z^2=c^2t^2
\end{equation}
而对于$\bar S$参考系有
\begin{equation}
  \bar x^2+\bar y^2+\bar z^2=c^2\bar t^2
\end{equation}
在欧氏空间中,无论坐标系如何变换距离都是不变的,而在四维时空(闵可夫斯基空间)中,上面的两个式子可以导出属于其的不变量
\begin{equation}
  x^2+y^2+z^2-c^2t^2=\bar x^2+\bar y^2+\bar z^2-c^2\bar t^2
\end{equation}
你会发现在闵可夫斯基空间中,事实上,这隐含着闵氏空间的线元在洛伦兹变换下是不变的,即
\begin{equation}
  \Delta s^2= (\Delta x)^2+(\Delta y)^2+(\Delta z)^2-c^2(\Delta t)^2=(\Delta \bar x)^2+(\Delta\bar y)^2+(\Delta\bar z)^2-c^2(\Delta \bar t)^2
\end{equation}
由于
\begin{equation}
  \Delta s^\mu=\Delta x^\mu
\end{equation}
但上述不变量中在4-矢量的第0分量出现了一个负号,这引导我们定义另一个4-矢量
\begin{equation}
  x_\mu=(-ct,x,y,z)=(-x^0,x^1,x^2,x^3)
\end{equation}
这称为协变矢量,而$x^\mu$称为逆变矢量,他们也可以用闵可夫斯基空间的度规相联系:
\begin{equation}
  a_\mu=\sum g_{\mu\nu}a^\nu
\end{equation}
其中度规
\begin{equation}
  g_{\mu\nu}=\mqty(\dmat{-1,1,1,1})
\end{equation}
有了这个定义,并定义两个4-矢量的内积为
\begin{equation}
  \sum a_\mu b^\mu=-a^0b^0+a^1b^1+a^2b^2+a^3b^3
\end{equation}
可以发现$\Delta s^2$实际上就是$\Delta s_\mu\Delta x^\mu$,即在闵氏空间中不变量为
\begin{equation}
  I=\Delta s_\mu\Delta s^\mu=-c^2(\Delta t)^2+(\Delta l)^2
\end{equation}
$I$被定义为两事件的间隔,其中$\Delta t$是在某个参考系中两事件的时间间隔,而$\Delta l$是该参考系中这两个事件发生的时间间隔。所以在狭义相对论中具有间隔不变性。

如果$I<0$,则时间占主导,称这个间隔是类时的;$I>0$则空间占主导,称这个间隔是类空的;$I=0$则有$\Delta l=ct$,时间和空间间隔由光速链接起来,则称这个间隔是类光的。事实上,对于类空和类光的事件,$I\leq 0\Rightarrow\Delta l\leq c\Delta t$即无论在哪个参考系下,两个事件发生的空间距离均允许一束光在晚些发生的事件发生前到达其发生的位置,这就构成了可以满足因果律的必要条件,所以满足因果律的两个事件一定是类时或类光的,但反之不然,不能说满足类时或类光的两个事件一定满足因果律。

借助时空图,我们可以在坐标系上标出物体在闵氏时空中的运动轨迹,对于$2$维时空(空间维度取作1),时空图的横坐标为$x$,纵坐标为$ct$
\begin{figure}[H]
\centering
  \includegraphics[width=7cm]{spacetime.png}
  \caption{时空图}
\end{figure}
在途中可以看到一个火箭在时空图中的运行轨迹,一个光子的运行轨迹和一个静止不动的粒子(他的轨迹是随着时间的变化$x$值不变),由于物体的运行速度不能超过光速,所以任何粒子的运动轨迹的斜率均大于光子。

\section{Relativistic Electrodynamics}
\subsection{The Transformation of Fields}
有了相对论的基本知识,我们可以回到电动力学,来探究电磁效应在狭义相对论下的变化情况了。(这里因为连贯性就没有介绍相对论力学了,这个也比较简单,如果有兴趣可以看相关教材)现在我们来考虑一个真空电容器(设图中视角为$S_0$系),
\begin{figure}[H]
\centering
  \includegraphics[width=5cm]{relem1.png}
\end{figure}
通过静电学知识可以获得,板间电场
\begin{equation}
  E_0=\frac{\sigma_0}{\epsilon_0}
\end{equation}
,现在我们在向$x_0$正方向以速度$v_0$运动的参考系$S$中看,此时因为洛伦兹收缩,在$S$参考系中,$l=l_0/\gamma_0=l_0\sqrt{1-v_0^2/c^2}$,而在相对论变换前后,板上带电量是一个不变量,所以在$S$参考系中,
\begin{equation}
  \sigma=\gamma_0\sigma_0
\end{equation}
相应的,$S$系中有
\begin{equation}
  E_y=\gamma_0\frac{\sigma_0}{\epsilon_0}=\gamma_0E_0
\end{equation}
当然,相对论变换只对运动方向维度有变化,相应的变化体现在电场中是垂直方向的变化,所以电场在相对论下有变换
\begin{equation}
  E^\perp=\gamma_0 E_0^\perp
\end{equation}
而平行于运动方向的电场强度不会改变
\begin{equation}
  E^\parallel=E_0^\parallel
\end{equation}
但这还没完,我们还没有考虑磁场,事实上,在S系中,有等效电流(因为在$S_0$系中静止的电荷在$S$系中看起来是运动的)
\begin{equation}
  \vb*{K}_\pm=\mp \sigma v_0\vb*{\x} 
\end{equation}
通过安培环路积分可以得到在两板中间有磁场
\begin{equation}
  B_z=\frac{1}{2}\mu_0 (K_+-K_-)=-\mu_0\sigma v_0
\end{equation}
所以在$S$系中,电场和磁场有$(9.38),(9.42)$的形式,但注意到$(9.42)$并没有完成变换,我们还需要找到另一个参考系,假设参考系$\bar S$以速度$v$沿着$x$正方向运动,则相对$S_0$系,其运动速度为
\begin{equation}
  \bar v=\frac{v+v_0}{1+vv_0/c^2}
\end{equation}
在$\bar S$系中,应当有
\begin{equation}
  \bar E_y=\frac{\bar \sigma}{\epsilon_0},\quad\bar B_z = -\mu_0\bar \sigma\bar v
\end{equation}
$S_0,\bar S$间有变换因子
\begin{equation}
  \bar \gamma=\frac{1}{\sqrt{1-\bar v^2/c^2}}
\end{equation}
则
\begin{equation}
  \bar E_y=\bar \gamma\frac{\sigma_0}{\epsilon_0}=\frac{\bar \gamma}{\gamma_0}E_y,\quad\bar B_z=-\frac{\bar \gamma}{\gamma_0}\mu_0\sigma\bar v
\end{equation}
下面来求$\bar \gamma/\gamma_0$
\begin{equation}
  \frac{\bar \gamma}{\gamma_0}=\frac{\sqrt{1-v_0^2/c^2}}{\sqrt{1-\bar v^2/c^2}}=\frac{c^2-v_0^2}{c\sqrt{1-\frac{c^2(v+v_0)^2}{(c^2+vv_0)^2}}}=\frac{(1+vv_0/c^2)\sqrt{c^2-v_0^2}}{\sqrt{(c^2-v_0^2)(1-v^2/c^2)}}=\frac{1+vv_0/c^2}{\sqrt{1-v^2/c^2}}
\end{equation}
受篇幅限制,这里就没有写的很详细了,计算这个公式可能需要一点耐心,读者如果在意数学推导可以自行计算,令
\begin{equation}
  \gamma=\frac{1	}{\sqrt{1-v^2/c^2}}
\end{equation}
于是
\begin{equation}
  \frac{\bar \gamma}{\gamma_0}=\gamma(1+vv_0/c^2)
\end{equation}
代回(9.46),可以得到
\begin{align}
\begin{cases}
		    \bar E_y=\gamma (E_y+\frac{vv_0}{c^2}\frac{\sigma}{\epsilon_0})=\gamma(E_y+\mu_0vv_0\sigma)=\gamma(E_y-vB_z)	\\
		    \bar B_z=-\gamma(1+vv_0/c^2)\mu_0\sigma\frac{v+v_0}{1+vv_0/c^2}=-\gamma(v+v_0)\mu_0\sigma=\gamma(B_z-\frac{v}{c^2}\frac{1}{\epsilon_0}\sigma)=\gamma(B_z-v/c^2E_y)
\end{cases}
\end{align}
不难想象,如果我们考虑下图的系统
\begin{figure}[H]
\centering
  \includegraphics[width=5cm]{relem2.png}
\end{figure}
可以得到变换
\begin{align}
\begin{cases}
		    \bar E_z=\gamma(E_z+vB_y)	\\
		    \bar B_y=\gamma(B_y+v/c^2E_z)
\end{cases}
\end{align}
而由于系统是朝着$x$方向运动,$x$方向的场不会改变(因为其正交方向均不会发生洛伦兹收缩)
\begin{align}
	  \begin{cases}
	  \bar E_x =E_x\\
	  \bar B_x = B_x	
	  \end{cases}	
\end{align}
至此,我们就得到了在狭义相对论下电磁场的变换形式。
\subsection{The Field Tensor}
其实不难发现,这个变换公式和洛伦兹变换很像,或许电磁场是满足洛伦兹变换的。只是我们之前考虑的洛伦兹变换是针对4-矢量的,他只有四个自由度,但是在这里我们的$\E,\B$均有三个自由度,我们需要能表征六个自由度的数学形式。首先能联想到的是一个六维向量,但可惜洛伦兹变换矩阵$\Lambda$是$4\times 4$的,只能作用在四维向量或有四行的矩阵上,这让我们联想到了可以使用一$4\times 4$的矩阵来表示电磁场,但这给出了16个自由度,显然我们不需要那么多,所以可以让这个矩阵有某些对称性,比如使用二阶反对称张量来表示,此时对角元均为0,非对角元有$t^{\mu\nu}=-t^{\nu\mu}$,这就消除掉了${4+6}$个自由度,剩余的6个自由度恰好可以表示我们的电磁场,一个四维的二阶反对称张量形式如下
\begin{equation}
t^{\mu v}=\left(\begin{array}{cccc}
0 & t^{01} & t^{02} & t^{03} \\
-t^{01} & 0 & t^{12} & t^{13} \\
-t^{02} & -t^{12} & 0 & t^{23} \\
-t^{03} & -t^{13} & -t^{23} & 0
\end{array}\right)
\end{equation}
他的洛伦兹变换需要两次求和
\begin{equation}
  \bar t^{\mu\nu}=\sum_\lambda\sum_\sigma\Lambda^\mu_\lambda\Lambda^\nu_{\sigma} t^{\lambda\sigma}
\end{equation}
注意到这似乎不是通常的矩阵乘法形式(因为第二项和第三项的乘法是某一行的所有列元素乘在一起),所以如果你觉得这个形式不是非常容易接受的话,也可以改写为矩阵乘法的形式
\begin{equation}
  \bar t^{\mu\nu}=\sum_\lambda\sum_\sigma\Lambda^\mu_\lambda t^{\lambda\sigma}\Lambda^\nu_{\sigma} =(\Lambda t \Lambda^T)^{\mu\nu}
\end{equation}
在对第二个$\Lambda$取转置以后这就成为了标准的矩阵乘法。现在我们来考虑场张量的洛伦兹变换的形式,为了阅读方便,我这里再写一遍$\Lambda$矩阵
\begin{equation}
\Lambda=
\left(\begin{array}{cccc}
\gamma & -\gamma \beta & 0 & 0 \\
-\gamma \beta & \gamma & 0 & 0 \\
0 & 0 & 1 & 0 \\
0 & 0 & 0 & 1
\end{array}\right)
\end{equation}

\begin{equation}
  \bar t^{01}=\Lambda_0^0\Lambda_1^1t^{01}+\Lambda_1^0\Lambda_0^1t^{10}=\gamma^2t^{01}+\gamma^2\beta^2t^{10}=\gamma^2(1-\beta^2)t^{01}=t^{01}
\end{equation}
\begin{equation}
  \bar t^{02}=\Lambda_0^0\Lambda^2_2t^{02}+\Lambda_1^0\Lambda_2^2t^{12}=\gamma (t^{02}-v/c t^{12})
\end{equation}
\begin{equation}
  \bar t^{03}=\Lambda_0^0\Lambda^3_3t^{03}+\Lambda_1^0\Lambda_3^3t^{13}=\gamma (t^{03}-v/c t^{13})
\end{equation}
\begin{equation}
  \bar t^{12}=\Lambda_0^1\Lambda^2_2t^{02}+\Lambda_1^1\Lambda_2^2t^{12}=\gamma(t^{12}-v/c t^{02})
\end{equation}
\begin{equation}
  \bar t^{13}=\Lambda_1^1\Lambda^3_3t^{13}+\Lambda_0^1\Lambda_3^3t^{13}=\gamma (t^{13}-v/c t^{03})
\end{equation}
\begin{equation}
  \bar t^{23}=t^{23}
\end{equation}
比较这六个变换式和(9.50,9.51,9.52),可以构造出两个满足洛伦兹变换的张量:$F^{\mu\nu},G^{\mu\nu}$,
\begin{equation}
F^{\mu \nu}=\left(\begin{array}{cccc}
0 & E_x / c & E_y / c & E_z / c \\
-E_x / c & 0 & B_z & -B_y \\
-E_y / c & -B_z & 0 & B_x \\
-E_z / c & B_y & -B_x & 0
\end{array}\right)
\end{equation}
\begin{equation}
G^{\mu \nu}=\left(\begin{array}{cccc}
0 & B_x & B_y & B_z \\
-B_x & 0 & -E_z / c & E_y / c \\
-B_y & E_z / c & 0 & -E_x / c \\
-B_z & -E_y / c & E_x / c & 0
\end{array}\right)
\end{equation}
当然你也可以用自然单位制(令$c=1$)让这个张量看起来更好看,不过这里还是取国际单位制。

\subsection{Electrodynamics in Tensor Notatioin}
\subsubsection{The Transformation of Sources}
以上我们已经得到了张量形式的电磁场的表示方法,接下来我们要给出张量形式下的Maxwell方程组。而由于我们要是在狭义相对论体系下在做这件事,场张量已经满足了洛伦兹协变性,但我们还没有定义关于源的,满足洛伦兹协变的矢量或者张量,下面来考虑源的变换。

在体系中,电荷量不变,但体积在运动方向的尺度由于洛伦兹收缩,在$\bar S$参考系下要缩小一个因子$\frac{1}{\gamma}$,于之前推出电磁场的变换类似,同样考虑$S_0,S,\bar S$系,于是
\begin{equation}
  \rho=\frac{Q}{V}=\gamma_0\frac{Q}{V_0}=\gamma_0 \rho_0,\quad \bar\rho=\frac{\bar\gamma}{\gamma_0}\rho
\end{equation}
\begin{equation}
  J_x=\rho v_0 
\end{equation}
所以同样的,有
\begin{equation}
  \bar J_x=\bar \rho \bar v=\frac{\bar\gamma}{\gamma_0}\frac{\bar v}{v_0}J_x
\end{equation}
在前面已经推出了
\begin{equation}
  \frac{\bar\gamma}{\gamma_0}=\gamma(1+vv_0/c^2)
\end{equation}
于是有变换关系
\begin{equation}
  \bar \rho=\gamma(\rho+v/c^2 J_x)
\end{equation}
\begin{equation}
  \bar J_x=\gamma(1+vv_0/c^2)\frac{v/v_0+1}{1+cc_0/c^2}\rho v_0=\gamma(1+v/v_0)\rho v_0=\gamma(J_x+v\rho)
\end{equation}
或者也可以改写为更明显的形式
\begin{align}
\begin{cases}
	c\bar \rho=\gamma(c\rho+\beta J_x)\\
	\bar J_x=\gamma(J_x+\beta c\rho)
\end{cases}
\end{align}
这正是洛伦兹变换的形式!!所以我们可以定义关于源的4-矢量
\begin{equation}
  J^{\mu}=(c\rho,J_x,J_y,J_z)
\end{equation}
下面来考虑电流的连续性方程:
\begin{equation}
  \div{\J}=-\pdv{\rho}{t}
\end{equation}
即
\begin{equation}
  \sum_{i=1}^3 \pdv{J^i}{x^i}=-\pdv{J^0/c}{x^0/c}=-\pdv{J^0}{x^0}
\end{equation}
这意味着
\begin{equation}
  \sum_i \pdv{J^i}{x^i}=0
\end{equation}
\subsubsection{Maxwell Equations}
下面我们来推导张量形式的麦克斯韦方程组。
首先写出三维欧氏空间下的形式
\begin{align}
        \begin{cases}
                \div{\vb*{E}}=\frac{\rho}{\epsilon_0}\\
                \curl{\vb*{E}}=-\pdv{\vb*{B}}{t} \\
                \div{\vb*{B}}=0\\
                \curl{\vb*{B}}=\mu_0\vb*{J}+\mu_0\epsilon_0\pdv{\vb*{E}}{t}& 
        \end{cases}
\end{align}
第一式可写为
\begin{equation}
 c \sum_{i=1}^3\pdv{F^{0i}}{x^i}=\frac{J^0/c}{\epsilon_0}\Rightarrow \sum_{i=1}^3\pdv{F^{0i}}{x^i}=\mu_0J^0
\end{equation}
事实上由于$F^{00}=0$,求和可以取遍0~3
\begin{equation}
  \sum_i\pdv{F^{0i}}{x^i}=\mu_0 J^0
\end{equation}
第四式的$\x$分量可写为
\begin{equation}
  \pdv{F^{12}}{x^{2}}+\pdv{F^{13}}{x^3}=\mu_0J^1+\mu_0\epsilon_0\pdv{cF^{01}}{x^0/c}=\mu_0J^1+\pdv{F^{01}}{x^0 }
 \end{equation}
 考虑到$F^{01}=-F^{10},F^{11}=0$上式可以写作求和
 \begin{equation}
  \sum_i\pdv{F^{1 i }}{x^i}=\mu_0 J^1
\end{equation}
对第四式的$\y,\z$做同样的工作并结合(9.78)可以推出
\begin{equation}
  \sum_\nu \pdv{F^{\mu \nu}}{x^\nu}=\mu_0 J^\mu
\end{equation}
这是非齐次的麦克斯韦方程组。通过$F^{\mu\nu}$得到非齐次的方程组(蕴含Maxwell方程组的2和3)也是可以的,但是形式上会略微复杂,这里还是使用其对偶张量$G^{\mu\nu}$,此时第三式可写为
\begin{equation}
  \sum_i \pdv{G^{0i}}{x^\i}=0
\end{equation}
第二式的$\x$分量为
\begin{equation}
  c\left(-\pdv{G^{12}}{x^2}+\pdv{G^{31}}{x^3}\right)=-\pdv{G^{01}}{x^0/c}
\end{equation}
考虑到$G^{11}=0,G^{12}=-G^{21}$,有
\begin{equation}
  \sum_i\pdv{G^{i1}}{x^i}=-\sum_i\pdv{G^{1i}}{x^i}=0\Rightarrow\sum_i\pdv{G^{1i}}{x^i}=0
\end{equation}
同样的,对$\y,\z$方向做同样的工作并结合(9.82),有
\begin{equation}
 \sum_\nu \pdv{G^{\mu\nu}}{x^\nu}=0
\end{equation}
这就是齐次的麦克斯韦方程组,如果借助爱因斯坦求和哑标,并使用$\partial_\nu=\pdv{}{x^\nu}$可以将其写成简洁的形式(虽然我本人就是很不喜欢这种写法...脑子转不过来哈哈哈 在脑子里还是要转换为求和,感觉还不如加上求和号写的清楚)
\begin{equation}
  \partial_\nu F^{\mu\nu}=\mu_0 J^\mu,\quad \partial_\nu G^{\mu\nu}=0
\end{equation}

\subsection{Potentials}
终于来到了最后一小节!在这一节中,我们要得到关于势的式子。

首先,根据之前的经验,由于标量势有一个自由度,而矢量势有三个维度,我们可以写出一个描述势的满足洛伦兹协变的4-矢量形式(虽然这并不是很直接)
\begin{equation}
  A^\mu=(V/c,A_x,A_y,A_z)
\end{equation}
在三维空间我们曾经推导出势和场的关系
\begin{equation}
  \E=-\nabla V-\pdv{\A}{t},\quad \B=\curl\A
\end{equation}
在张量形式下可以重写为
\begin{equation}
  F^{\mu\nu}=\pdv{A^\nu}{x_\mu}-\pdv{A^\mu}{x_\nu}
\end{equation}
注意这里的微分是针对协变矢量的。考虑麦克斯韦方程组中的非齐次方程$\sum_\nu \pdv{F^{\mu \nu}}{x^\nu}=\mu_0 J^\mu$,代入有
\begin{equation}
  \sum_\nu \left(\pdv{}{x_\mu}\pdv{}{x^\nu}A^\nu-\pdv{}{x_\nu}\pdv{}{x^\nu}A^\mu\right)=\mu_0 J^\mu
\end{equation}
这个式子比较复杂,我们可以来简化,一种简化的方法就是令
\begin{equation}
  \sum_\nu \pdv{}{x^\nu}A^\nu=0
\end{equation}
事实上,这意味着
\begin{equation}
  \div{\vb*{A}}+\pdv{V/c}{(ct)}=\div{\A}+\frac{1}{c^2}\pdv{V}{t}=0
\end{equation}
这实际上就是我们之前的洛伦兹规范!在这个规范下,有
\begin{equation}
  \sum_\nu \pdv{}{x_\nu}\pdv{}{x^\nu}A^\mu=\mu_0 J^\mu
\end{equation}
可以令
\begin{equation}
 \square^2=\pdv{}{x_\nu}\pdv{}{x^\nu}=-\frac{1}{c^2}\pdv[2]{}{t}+\nabla^2
\end{equation}
这就是我们之前给出的达朗贝尔算符,此时式(9.93)可以改写为
\begin{equation}
  \square^2 A^\mu=\mu_0 J^\mu
\end{equation}
至此,我们把初等的全部的电磁理论归结到了这样一个简单的方程,不得不说,这个公式非常令人振奋!

到这里,电动力学(I)就告一段落了,为什么我称它是电动力学I呢?是因为Griffiths的书虽然语言非常通俗、亲切,强调物理图像,但毕竟内容太少,可能在某种角度来看,这大概是电磁学+电动力学与应用初步+相对论电动力学导论哈哈哈,如果我后面有时间且对其他一些内容感兴趣的话,或许会以专题形式在进行更新,但...这也是后面的事啦!这篇复习总结的篇幅比我想象的多得多,有可能是我没有把握住节奏,但是确实有很多东西想说哈哈哈,尤其是相对论部分,我感觉Griffiths偏向于告诉你这应该是什么,从这个形式怎么推导回原来熟悉的形式,但是我可能更喜欢从旧知识到新知识的推导嘿嘿,所以加了一些自己的想法,如果有错误也请告知!

电动力学在物理图像上是很优美的(这和统计物理有的一比哈哈),但是两者需要在数学上下的功夫也不少(捂脸),嗯...我的数学就不太好,所以我就有只喜欢公式的推导但对做习题比较厌烦。不过曾经有一个老师告诉我,理论物理不做题等于没学!所以读者如果是物理系的学生,希望学好四大力学,打下理论物理的基础的话可能还是要多做题,虽然到这个阶段,大多数题目和题目的解是ugly的以至于你可能花了很久很久因为一些数学上的小失误导致无法得到一个正确的答案,但是还是不要做“美好形式的奴隶”。







\end{large}
%\end{CJK}

\end{document}